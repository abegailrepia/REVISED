\documentclass[12pt]{cspcccsthesis}
% preamble

\title{Comparative Analysis of Discrete-Event And Continuous Simulation Models For Students Flow Optimization}
\authorOne{Regaspi, May Angeline M.}
\authorTwo{Repia, Abegail C.}
\authorThree{Ruin, Joy Ann M.}
\degree{Bachelor of Science in Computer Science}
\approvaldate{December 5, 2025}
\school{Camarines Sur Polytechnic Colleges}
\adviser{KAELA MARIE N. FORTUNO, MIT}
\dean{ROSEL O. ONESA, MIT}
\committeeMemberOne{FREDDIE B. PRIANES, MIT}
\committeeMemberTwo{JOSEPH JESSIE S. OÑATE, MSc}
\committeeChair{TIFFANY LYN O. PANDES, MSc}
\department{College of Computer Studies}
\thesisAbstract{This study assesses the effectiveness of Discrete-Event Simulation (DES) versus Continuous Simulation (CS) in optimizing student flow at the Registrar’s Office of Camarines Sur Polytechnic Colleges (CSPC). The research aims to address issues like long queues and service delays by using operational data to build and evaluate DES and CS models according to metrics like waiting time and system efficiency. The findings indicate that DES outperformed CS in operational capacity and resource utilization, achieving a higher throughput of 21.9 students per hour compared with 12.6 students per hour for CS, a resource utilization rate of 94.3\% compared with 49.5\%, and greater overall system efficiency of 34.5\% compared with 19.5\%. The study concludes that DES is better for short-term optimization, whereas CS is more appropriate for long-term trend analysis. Additionally, a 3D prototype system was developed to aid in the visualization and optimization of student service workflows.}
\keywords{Discrete-Event Simulation, Continuous Simulation, Student Flow Optimization}

% Load glossary entries
\input{settings/glossary.tex}

% document body
\begin{document}

\makeTitlePage{December}{2025}

\begin{frontmatter}
    \input{approvalPage}
    \makePanelofExaminers{}
    \makeDedication{The unwavering love and support of our families, especially our parents, have been the foundation of our journey and the inspiration behind this achievement.

    Guidance and encouragement from our professors and mentors greatly shaped this research, and their dedication to our growth is deeply appreciated.

    Supportive friends made the challenges more bearable, offering comfort and motivation throughout the process.

    This work also reflects the teamwork, perseverance, and shared commitment within our group—a milestone we proudly accomplished together.}
    
    \begin{acknowledgments}

The researchers gratefully acknowledge the support and guidance of all individuals who contributed to the completion of this study. We sincerely thank \textbf{Ms. Rosel Onesa}, OIC Dean, College of Computer Studies, for her encouragement and administrative support; \textbf{Ma'am Kaela Marie Fortuno}, our thesis adviser, for her guidance and valuable insights; \textbf{Mr. Juan Rodrigo Badiola}, our consultant, for his expert advice; and \textbf{Ms. Jayvie Margate}, our grammarian, for her careful review and assistance in improving the quality of this manuscript.

We also extend our gratitude to the honorable panelists, \textbf{Ma'am Tiffany Lyn O. Pandes}, \textbf{Sir Freddie B. Prianes}, and \textbf{Sir Joseph Jessie S. Onate}, for their constructive comments and helpful suggestions during the defense. We likewise thank \textbf{Ms. Marri Grace Morata}, the Secretary, for her assistance and encouragement, and the Registrar’s Office for their prompt support in providing the necessary documents for this study. Their cooperation and professionalism greatly contributed to the smooth completion of our academic requirements. The insights and logistical assistance they provided strengthened the overall quality and credibility of this research.

Above all, we thank the researchers’ loving parents for their unwavering love, support, and sacrifices throughout this journey. Their constant encouragement, guidance, and motivation inspired us to persevere through challenges and remain focused on our goals. Most importantly, we offer our sincerest gratitude to Almighty God for the wisdom, strength, and guidance that made this study possible. His blessings sustained us throughout the entire research process.


\end{acknowledgments}
    \makeAbstract
    \makeTOC
    \makeListOfTables
    \makeListOfFigures
\end{frontmatter}

\begin{thesisbody}
    \chapter{Introduction}
\begin{refsection}

This chapter introduces the problem investigated in this research, providing the necessary background to identify the goals of the project. The chapter will present the problem identified, the research targets resulting from it, and a written project summary in detail. In addition, it describes the meaning, scope, and limitations of the project, delineates the target beneficiaries, and features a glossary of key terms used in this study.

\section{Background of the Problem}

The efficient flow of people within educational institutions, particularly in dynamic environments like colleges, schools, and programs (CSP), is crucial for maintaining operational effectiveness and enhancing the student  experience. Bottlenecks, long waiting times, and inefficient processes could lead to frustration, decreased productivity, and ultimately impact student satisfaction and academic success. Optimizing these flow-of-people patterns requires a comprehensive understanding of the underlying processes and the ability to test and evaluate potential improvements. Simulation modeling offers a powerful approach to analyze and optimize these complex systems and allowed administrators to make data-driven decisions regarding resource allocation, process redesign, and infrastructure improvements. Two prominent simulation techniques, Discrete-Event Simulation (DES) and Continuous Simulation (CS), offered distinct advantages and disadvantages in modeling student flow within a CSPC, particularly within critical service areas such as the Registrar’s Office, which was a pivotal point in the school \cite{jamil2023optimizing}.

Discrete-Event Simulation (DES) models systems as a sequence of events occurring at discrete points in time. This approach was particularly well-suited for modeling systems with 2 distinct entities (e.g., students) that moved through a series of processes (e.g., registration, payment, document submission). DE simulation excels at capturing the variability and stochasticity inherent in many real-world systems, such as variable arrival rates, service times, and resource availability \cite{do2022metamodel}. This level of granularity could provide valuable insights into the impact of queues, resource utilization, and staffing level on student flow. In contrast, Continuous Simulation (CS)  models systems using differential equations that described the continuous changes in state variables over time. While less common in modeling individual student behavior, CS could have been useful for representing aggregate flow and systemic trends, such as the overall density of students in a particular area or the rate at which students completed a specific process. The choice between DES and CS depended heavily on the specific objectives of the simulation study and the level of detail required to address the researched questions. This study focuses on a CSPC, highlighting the need for targeted strategies to alleviate congestion and improve student service outcomes \cite{DigitalTwin2025}.

The Registrar’s Office handles functions from enrollment to graduation and represents a key area where optimized student flow could significantly impact efficiency and satisfaction. The general goal of this research is to comparatively analyze the application of DE and CS models in simulating and optimizing student flow specifically within the Registrar’s Office setting \cite{Fonseca2023}. Our aim was to assess the strengths and limitations of each approach, identify bottlenecks, evaluate potential solutions, and ultimately improve the student experience. Recent studies have emphasized the importance of leveraging simulation modeling to enhance operational efficiency in higher education institutions. However, a comprehensive comparative analysis of DE and CS approaches, particularly in the context of the Registrar’s Office within a CSPC, remain limited. This research aimed to bridge this gap by providing a rigorous comparison of these two simulation techniques, offering practical guidance for administrators seeking to optimize student flow and enhance service delivery. By evaluating the effectiveness of DE and CS, the study will contributed for better understanding of how these methods can be applied to improve the overall educational experience in the context of a CSPC. 

\section{Statement of the Problem}

The Registrar’s Office faces operational inefficiencies in student flow management, which result in extended wait times and high congestion during busy times. The office faced challenges because it lacks dynamism and modeling tools that would simulate real-world registration scenarios to optimize service processes effectively. The combination of Discrete Event and Continuous Simulation models would allow for thorough student flow pattern analysis to reveal essential metrics about queue lengths and wait times and resource utilization, which would have helped identify operational bottlenecks and enhanced efficiency.The evaluation of these simulation models against key metrics such as average waiting time, queue length, service utilization, throughput, and overall system efficiency was necessary to determine the most effective approached for the Registrar’s Office. The comparison of these two simulation models would help decision-makers to select the appropriate model that balances accuracy and computational efficiency. This evaluation would support data-driven improvements in scheduling, staffing, and process design, ultimately enhancing student satisfaction and resource management.

Moreover, the creation of a 3D-based prototype system to visualize the student flow design transformed dry simulation data into an engaging and experiential journey for stakeholders. Visualization tools like these made it easier to communicate intricate operational scenarios more clearly, allowing administrators to explore spatial designs and test interventions virtually before actual implementation. By embedding simulation outputs within an interactive 3D environment, the Registrar’s Office enhanced decision-making, minimized implementation risks, and accelerated workflow execution, resulting in more effective and efficient student service delivery.

\section{Objectives of the Study}
This study focuses on optimizing student flow processes within the Registrar’s Office of Camarines Sur Polytechnic Colleges (CSPC). By comparing discrete-event and continuous simulation models, we aim to identify bottlenecks, evaluate potential process improvements, and ultimately enhance the efficiency and effectiveness of the Registrar’s Office in serving its students.


\subsection{General Objective}
The general objective of this study was to conduct a Comparative Analysis of Discrete-Event and Continuous Simulation models to identify and evaluate optimal strategies for improved student flow efficiency and service quality within the Registrar’s Office of Camarines Sur Polytechnic Colleges.

\subsection{Specific Objectives}More specifically, this study aimed to:

\begin{enumerate}
    \item Implement Discrete-Event and Continuous Simulation models to analyze student flow in the Registrar’s Office.
    
    \item Evaluate the performance of the simulation models based on the following key metrics:

    \begin{enumerate}
        \item[a.)] average waiting time
        \item[b.)] queue length
        \item[c.)] service utilization
        \item[d.)] throughput
        \item[e.)] system efficiency
    \end{enumerate}
    
    \item Develop a 3D prototype system that visualizes student flow to support data-driven decision-making and enhance operational planning.
\end{enumerate}
   

\section{Significance of the Study}

The results of this study were beneficial for the following:\begin{itemize}    \item \textbf{Camarines Sur Polytechnic Colleges (CSPC):} This study would provide the CSPC with a data-driven understanding of its current student flow processes. Comparison of DES and continuous simulation would allow CSPC to identify bottlenecks, evaluate possible interventions, and implement strategies that optimize resource utilization and reduce student wait times, and ultimately enhance the overall student experience. By using the findings, CSPC could improve its operational efficiency and maintain a competitive edge in the education sector.
\end{itemize}

\begin{itemize}    
\item \textbf{Registrars:} The study offered registrars valuable insights into the effectiveness of service workflows and allowed them to improve queue management, enhance service delivery, and make informed decisions during high-demand periods such as enrollment and registration.
\end{itemize}     
     
\begin{itemize}
\item \textbf{Students:} The goal of the study was to streamline student flow, make it easier for them to access courses, reduce overcrowding, and cut down on waiting time, which results in a more positive educational experience and smoother academic progress. This improvement contributes to a more organized and student-centered environment that supports academic success and well-being.

\end{itemize}

\begin{itemize}
\item \textbf{Faculty and Academic Staff:} It offered valuable insights for refining class schedules, minimizing course conflicts, and maximizing the use of classrooms and teaching resources, ultimately enriching both the teaching and learning experience. The study also supports more informed academic planning by identifying areas where scheduling and resource allocation can be further optimized.
\end{itemize}

 \begin{itemize}    
 \item \textbf{CSPC Administrators and Planners:} This study aids decision-makers at CSPC in choosing the best simulation model to enhance student flow, optimize resource allocation, improve scheduling, and boost overall operational efficiency.
 \end{itemize}     
 
\begin{itemize}    
 \item \textbf{Researchers:} As the researcher, this study lies in its potential to generate valuable insights and contribute to academic knowledge. It could also be utilized as a stepping stone toward further research and investigations to improve understanding and use of the Discrete-Event Simulation (DES) and Continuous Simulation (CS) models.
 \end{itemize}
     
\begin{itemize}    
\item \textbf{Future Researchers:} This research can be an excellent source of reference for future researchers because it also helped in conducting related studies and improvements of the Discrete-Event and Continuous Simulation Model.
\end{itemize}

\section{Scope and Limitation}

This study was specifically centered on the Camarines Sur Polytechnic Colleges (CSPC), particularly its Registrar’s Office, where we addressed the CSPC administrators and students, and a comparative analysis of Discrete-event Smulation and Continuous Simulation models. Current student flow processes at CSPC Registrar’s Office were optimized to improve student flow, reduce waiting time, and enhance satisfaction by developing, validating, and comparing simulation models that used real-world data from the Registrar’s Office. The core objective was to determine which simulation approach—DES or CS—provides a more accurate, efficient, and practical solution for improved student flow and related administrative processes within the unique context of CSPC, ultimately aimed at streamlining operations and elevating service quality for both students and staff.

A critical limitation of this study is the reliance on data availability and the inherent variability of student behavior at CSPC. The accuracy and effectiveness of both Discrete-Event and Continuous Simulation models were highly dependent on the quality, completeness, and representative of the data collected from the Registrar’s Office. However, historical data might have been incomplete or failed to account for recent changes in student demographics, academic policies, or technological advancements, potentially limiting the models’ predictive power. Moreover, student behavior—such as arrival patterns, service requested, and processing times—is subject to significant variation due to factors like academic deadlines, personal circumstances, and evolving administrative procedures. This unpredictability introduces substantial uncertainty into the simulation results, making it challenging to fully capture the complexity and dynamism of real-world student flow. As such, while the models provided valuable insights for process optimization, their recommendations must be interpreted cautiously, with a clear understanding of the risks posed by data limitations and the inherent variability in student behavior.

\section{Project Dictionary}
The Project Dictionary contains the technical terms that defined the conceptual and operation of this study:

    \begin{itemize}   
    \item \textbf{Arrival Rate.} In Collins Dictionary, it is defined as the number of arrivals that occur within a specific time interval. The frequency at which students or service seekers enter the Registrar’s Office per unit of time, typically measured in arrivals per hour or day \cite{hassin2023strategic}. In this study, arrival rate was a key parameter for both DES and CS models, helping to simulate and analyze patterns of student flow and demand.
    
    \item \textbf{Comparative Analysis.} A method of evaluating two or more approaches, models, or systems by systematically comparing their features, performance, or outcomes \cite{study2025comparative}. In this study, comparative analysis is used to assess the effectiveness of DES and CS models in optimizing student flow within the Registrar’s Office at CSPC.
   
    \item \textbf{Continuous Simulation (CS).} A modeling approach where system variables change continuously over time, typically described by differential equations \cite{ElsevierContinuousSimulation2025}. 
    In this study, CS is used to represent system dynamics like arrival trends and resource usage over extended periods.
    
    \item \textbf{Discrete-Event Simulation (DES).} A modeling technique that represents a system as a sequence of events occurring at specific points in time \cite{felipe2022}. In this study, DES is applied to model individual student transactions and interactions at defined
    service stages.

    \item \textbf{First-Come, First-Served (FCFS).} A queue discipline where entities are served in the exact order they arrived \cite{kumar2025modeling}. In this study, the method is applied in both simulation models to reflect the typical service policy at the Registrar’s Office.

      \item \textbf{Optimization.} The process of making changes to maximize system performance \cite{downey2022optimization}. In this study, optimization focuses on improved workflows and reduced delays through simulation-based analysis.

     \item \textbf{Queue Length.} The number of entities (e.g., students) waiting in line for service at any given time \cite{anuruddhika2022approaches}. In this study, queue length served as a key performance metric in both Discrete-Event Simulation (DES) and Continuous Simulation (CS) models used to assess congestion levels and identify potential service bottlenecks in the Registrar’s Office.

     \item \textbf{Registrar’s Office.} The administrative unit responsible for academic record-keeping, registration, and document issuance \cite{metto2022study}. In this study, offices serves as the setting where flow optimization is applied.

      \item \textbf{Service Time.} The duration required to complete a student’s transaction at a particular service station \cite{vu2021service}. In this study, incorporates statistical distributions to simulate service time variability.

    \item \textbf{Service Utilization.} The percentage of time service resources (e.g., staff, counters) are actively engaged in serving students \cite{Kumar2025_ResourceUtilization}. In this study, a high utilization rate suggests efficient use of resources, while a low rate may indicate underutilized or overstaffed resources.
    
    \item \textbf{Simulation Model.} A virtual representation of a real-world process or system used to observe performance under varying conditions \cite{duran2020simulation}. In this study, it uses simulation models to experiment with and evaluate student flow scenarios in the Registrar’s Office.
    
    \item \textbf{Student Flow.} A flow was commonly understood as the smooth and continuous movement of people or processes from one point to another.
    In research, student flow is described as the sequence of steps students follow when accessing services \cite{aye2020student}.
    In this study, this model process to identify bottlenecks, reduce congestion, and improve throughput, ultimately leading to a more efficient system for students and administrators.

    
    \item \textbf{System Efficiency.} A measure of how effectively the system uses its resources to achieve desired outcomes with minimal waste \cite{Oliveto1998}. In this study, included balanced staff workloads, reduced idle time, and improved student satisfaction.


    \item \textbf{Throughput.} The number of students successfully served within a specified period \cite{subramaniyan2021artificial}. In this study, this metric is used to evaluate the overall capacity and performance of the simulation models. In this study, higher throughput
    indicates better system efficiency and reduced service delays, making it a key indicator in assessing process effectiveness.
    
\end{itemize}


%=======================================================%
%%%%% Do not delete this part %%%%%%
\clearpage

\printbibliography[heading=subbibintoc, title={\texorpdfstring{\centering}{} Notes}]
\end{refsection}
     \chapter{Related Literature and Studies}
\begin{refsection}

Following the thorough review of the researcher, this chapter included reviews of relevant articles and then the literature. It concentrated on a number of areas that aided in the development of this research. The state-of-the-art synthesis and the discrepancy between the study’s similarities and differences with comparable literature researched was further examined.


\subsection{Student Flow Optimization in Educational Institutions}

Student flow optimization refers to the strategic organization and enhancement of student-related processes within educational institutions, ensuring timely and efficient movement through administrative or academic systems such as registration, advising, and enrollment. Efficient flow leads to reduced wait times, balanced workloads for administrative staff, and improved student satisfaction. The registrar’s office, being central to managed academic records, enrollment, and certification, is a critical target for these optimizations. By refining how students interact with this office, institutions can significantly improve overall operational efficiency. 

The digital transformation of higher education has accelerated significantly following the COVID-19 pandemic. According to Modern Campus [\citeyear{moderncampus2025sms}], approximately 70\% of North American educational
institutions have now adopted cloud-based student information systems, with adoption rates
continuing to climb as institutions recognize the operational and financial benefits of these modern
platforms. This transformation encompasses not only technological upgrades but also fundamental
changes in how student services are conceptualized and delivered. Cloud-first architectures provide
the scalability and flexibility that modern higher education demands, allowing institutions to rapidly
adjust system capacity based on enrollment fluctuations, peak usage periods, and evolving functional
requirements. Mobile accessibility has become a non-negotiable feature, with students expecting to
access academic information, register for courses, and communicate with institutional services
through their smartphones and tablets. This shift toward comprehensive digital ecosystems requires
robust models to understand and optimize the underlying workflows that support student services.

Recent research underscores the importance of robust queue modeling and simulation in achieving these goals. \citeauthor{sevin2025analysis} [\citeyear{sevin2025analysis}], provided a comprehensive analysis of queue models in simulation applications, highlighting how the selection of appropriate queuing frameworks can directly impact the efficiency of student services. Their findings support the need for data-driven, simulation-based strategies to manage student flow, particularly in high-traffic administrative settings. Several studies reinforce this perspective. For instance, \citeauthor{falolo2022student} [\citeyear{falolo2022student}], emphasized the significance of digitizing student registration systems to improve recorded management and process responsiveness. As student populations grow, manual processes often become bottlenecks, prompting the necessity for systems-driven optimization approaches.
 
Modern institutions are under pressure to minimize bureaucratic delays to enhance their service
delivery and competitiveness. As highlights by \citeauthor{delarue2021optimizing} [\citeyear{delarue2021optimizing}], operational challenges in academic institutions are pervasive, and algorithmic and system-based approaches are proposed to streamline
administrative operations. In practice, student flow issues may manifest as long queues,
underutilized personnel, or misallocated resources. Addressing these inefficiencies requires accurate modeling of processes and experimentation with potential improvements, which simulation technologies make possible. As \citeauthor{Biazen2025} [\citeyear{Biazen2025}] noted, the verification and validation of simulation
models are critical to ensure that proposed optimizations are both reliable and effective, especially
when applied to dynamic and complex environments like educational institutions. 

Ultimately, student flow optimization was not just a technological endeavor—it involves institutional changes, data-driven decisions, and the adoption of system modeling tools that simulate various policy implementations before real-world deployment. This provides a low-risk environment for testing interventions that can potentially save time, reduce cost, and elevate the student experience.


\subsection{Flow Simulation}

Flow simulation was a significant aspect that influences operational processes in service settings, like the Registrar’s Office at CSPC. Discrete-Event Simulation (DES) provides a representation of a system as a series of events that happen in time. DES was especially useful in studying queue systems and throughput of resources in environments where different transactions, like student registration, occurred irregularly. Continuous Simulation (CS) provides a continuous change in the system over time. Unlike DES, which represents distinct moments in time, CS captures flow with the ability to observe system dynamics between events. Each simulation method offered different distinct observations and insights that can be used to inform improved student flow. It is noticeable that few have compared the effectiveness of DES and CS in an academic administrative context.

Recent research suggests both strengths and limitations of DES and CS in analyzing complex flow optimization problems. DES is ideal for modeling specific event-driven processes, e.g., arrivals of students, duration of service time, and queue management, which play a critical role in exploring bottlenecks and modeling operating scenarios for the Registrar’s Office. Continuous Simulation plays an important role in examining overall system trends and continuous resource interaction, which was useful for planning and capacity analysis. There have been many advancements in discrete-event models that can be integrated with continuous-time systems that would increases the fidelity and utility of the simulation to represent real-life processes. These types of comparisons are important for determining which type of modeling provides the best balance of accuracy, computational efficiency, and applicability in complex flow optimization for CSPC.

In addition, combining flow simulation with visualization applications and 3D prototypes has emerged as a valuable decision-support tool for process optimization. Systematic reviews indicate that integrating discrete event simulations (DES) with extended reality (XR) and immersive visualization is an effective means of encouraging stakeholders to understand processes by simulating scenarios before full-scale implementation—a benefit in itself. For CSPC’s Registrar’s Office, a 3D-based simulation prototype can represent invisible data as intuitive spatial representations, allowing the administrative team to understand where to accommodate congested students and review their layouts with a view to optimizing new routes or their entire base room. Combining simulation with traditional research advances what has emerged in simulation research that combines rigorous analytics with interactive visualization to assist in operational planning and delivery of services.


\subsection{Discrete-event simulation (DES) in Academic Operations}

Discrete-event simulation (DES) models systems as a sequence of distinct events that occur over time. Each event marks a change in system state—like a student arriving at the registrar, submitting documents, or being served. DES was particularly effective for analyzing queue systems and resource constraints, making it ideal for registrar processes involving large, varied student traffic and service types. 

According to the study conducted by \citeauthor{Modelling2021} [\citeyear{Modelling2021}], stochastic DES modeling demonstrates
practical effectiveness in optimizing operational throughput, a principle directly relevant to student
registration and service scenarios. Their work illustrates how DES can be used to identify
bottlenecks, improve throughput, and inform resource allocation strategies in complex, event-driven
systems. The application of discrete-event simulation in educational settings has become increasingly
sophisticated. A study by \citeauthor{droscher2021approach} [\citeyear{droscher2021approach}] demonstrated how Discrete-Event Simulation can be
effectively used to optimize university study programs by simulating various components such as
course demand, student enrollment choices, and scheduling conflicts. Their approach provided
valuable insights into the operational structure of academic institutions and revealed how different
scenarios can be tested virtually to predict their impact on system performance. For instance, by
modeling how students select courses and how these choices interact with capacity limits and
prerequisite structures, administrators can proactively adjust schedules or program offerings to
minimize course bottlenecks and delays in student progression.
 
Similarly, \citeauthor{fernandez2021resources} [\citeyear{fernandez2021resources}], applied DES to improve resource allocation in the context of educational service planning. Their simulation framework enables institutions to visualize and
forecast service level bottlenecks—such as those encountered in registration offices, advising
sessions, or lab access—and to test alternative resource distribution strategies. Through these
methods, educational organizations are empowered to reduce student wait times, balance staff
workloads, and enhance the overall efficiency of academic service delivery. Recent case study research
has reinforced these findings. \citeauthor{marsudi2020modeling} [\citeyear{marsudi2020modeling}], illustrated DES in action through the use of ARENA software to simulate student registration, achieving notable improvements in process efficiency.
The use of Arena simulation software has become standard practice in analyzing registrar processes,
as documented by both academic and practitioner sources.

DES also plays a crucial role in long-term planning. Institutions can simulate academic policy changes, like introducing a new curriculum or revising the credit load per semester, and assess how these changes affect student progression and graduation timelines. This makes DES not only a tool for operational management but also a valuable asset for policy development. With increased pressure to
maximize institutional effectiveness with limited budgets, DES helps ensure that academic operations
are both responsive and evidence-based. Furthermore, recent innovations in AI-assisted DES platforms
have expanded the capabilities of traditional simulation. \citeauthor{Fei2025AIassistedDES}  [\citeyear{Fei2025AIassistedDES}], proposed a framework of an
AI-assisted intelligent DES platform that leverages AI agents to transform conventional simulation
workflows. By automating data collection, preprocessing, and simulation model construction, as
well as analyzing and optimizing simulation results, the platform transforms conventional simulation
workflows. This integration of artificial intelligence with discrete-event simulation opens new
opportunities for more adaptive and responsive educational simulations.

\subsection{Continous Simulation (CS) for Systematic Process Modeling}

Continuous simulation (CS) differs from DES in that it models systems where change was constant over time rather than event-driven. This approach was useful in representing aggregated flows, such as the overall rate of student registration or service usage over a semester. CS often uses differential equations to model trends, making it effective for strategic, long-term planning rather than specific process bottlenecks. A study by \citeauthor{Biazen2025} [\citeyear{Biazen2025}], highlights that rigorous verification and validation are essential in Continuous Simulation (CS), especially when modeling complex systems such as those found in educational institutions. In CS, system variables evolve continuously over time, making it well- suited for representing long-term trends like resource utilization, arrival rates, and service demand. This approach allows administrators to analyze how changes in one area affect the entire process over an extended period. Ensuring model accuracy through validation helps simulate realistic system behavior, which is vital for strategic planning and process improvement.

In registrar operations, CS provides insight into cumulative impacts on student flow and reveals inefficiencies that might have been overlooked by discrete-event methods. It supports continuous monitoring and forecasting under various scenarios, enabling more informed decision-making.
According to the study of \citeauthor{salloum2020forecasting} [\citeyear{salloum2020forecasting}], continuous simulation was valued in dynamically updating models in real-time, which is useful for institutions dealing with ongoing changes in student
population, staffing, or service demand. CS enables smoothed approximation of processes like enrollment rates, document processing over weeks, and resource fatigue, which is less amenable to discrete modeling. As proposed in the study by \citeauthor{loossens2021comparison} [\citeyear{loossens2021comparison}], continuous models could accurately predict effective trends in processes, suggesting their value in understanding student behavioral trends and institutional responsiveness.

System Dynamics, a specific form of continuous simulation, has been applied broadly to educational
contexts. \citeauthor{groff2013dynamic} [\citeyear{groff2013dynamic}], demonstrated the application of System Dynamics modeling to analyze
educational systems, understand their complexity, and guide policy and system design. By mapping
the feedback structure of a system, system dynamics helps explain why a system behaves the way it is
and enables policymakers to test and plan for policies before implementing them. This methodology has
proven valuable for understanding not just the immediate operations but the long-term implications of
policy decisions. More recently, \citeauthor{Ballard2020CommunityBased} [\citeyear{Ballard2020CommunityBased}], described community-based system dynamics
(CBSD) as a participatory approach for engaging communities in understanding and changing complex
systems. This approach has been applied to school health and other educational initiatives,
demonstrating that system dynamics modeling can enhance collaboration, analysis, and community
action at multiple levels.

Another useful application of CS in education involves monitoring and improving student wellness and
retention. Variables such as mental health, study load, and extracurricular participation can be modeled
over time to identify when students are likely to disengage or under perform. Administrators could use
these insights to intervene early. This makes CS a valuable tool not only for academic success but also
for supporting student well-being. Though CS may lack the granularity of DES in handling discrete,
immediate actions, it excels in providing strategic insights over extended periods. Its continuous data
flow allows for the identification of slow-moving issues that DES might overlook, such as the
compounding effects of budget constraints on faculty workload. As institutions increasingly adopt
data-driven strategies, continuous simulation can support long-term initiatives like curriculum reform,
faculty development, and academic policy evaluation.

\subsection{Comparative Studies on DES and CS}
 
Comparing DES and CS requires evaluating their suitability based on system characteristics, goals, and available data. DES excels in systems with clear individual events and queue behaviors, while CS preferred for continuous, high-level flows. Both approaches have strengths and weaknesses, and the choice depends on whether micro-level detail or macro-level trends is more critical.

In the study of \citeauthor{Naciri2024ModelingSimulation}  [\citeyear{Naciri2024ModelingSimulation}], conducted a systematic review analyzing various simulation approaches, it was revealed that Discrete-Event Simulation (DES) was widely adopted in higher education due to its precision in modeling detailed administrative operations such as registration, enrollment, and academic advising. These processes are often governed by start-stop events and queued dynamics, making DES ideal for capturing their structure and behavior. The study emphasizes DES’s effectiveness in enhancing service delivery and resource allocation, particularly in high-demand offices like the registrar. The same review also identifies scenarios where Continuous Simulation (CS) provides superior modeling capabilities. CS is especially useful for representing broad institutional processes, such as budget forecasting, long-term faculty workload distribution, or energy consumption patterns, where changes evolve smoothly over time rather than through discrete steps.

This distinction in application is echoed in the work of \citeauthor{simio2025differences} [\citeyear{simio2025differences}], which clearly outlines how DES
excels in systems driven by event sequences and queue structures, while CS supports systems with
continuous, uninterrupted flow. Model selection should align with both system dynamics and
research objectives. Institutions can determine the most appropriate tools for analysis and planning by
understanding these distinctions. Comparing DES and CS requires evaluating their suitability based on
system characteristics, modeling goals, and available data. DES is highly effective when analyzing
discrete events, such as queued wait times, student check-ins, or processing duration, making it ideal
for micro-level administrative improvements. In contrast, CS is better suited for capturing macro-level
trends that unfold continuously over time, such as institutional budgeting or campus-wide resource
usage.

A practical example of DES application in educational registrar operations was documented by
Muhammad and colleagues [\citeyear{marsudi2020modeling}], who demonstrated that modeling and simulation using ARENA is
very useful in analyzing the registration process of new students. Their work on student enrollment
processes confirms that DES can effectively identify inefficiencies and recommend improvements
using performance metrics of waiting times and queue lengths. More recent work by researchers like \citeauthor{pang2025des_enrolment} employed DES approaches for efficient student enrollment, contributing to the body of knowledge
regarding DES applications in the educational sector and serving as practical strategies for improving
student service operations [\citeyear{pang2025des_enrolment}].

Both methodologies offer distinct advantages and limitations. Therefore, the decision to use one over
the other should be grounded in whether the focus is on individual interactions and event precision or
on long-term patterns and aggregate system behavior. The comparative analysis of Discrete-Event and
Continuous Simulation models offers valuable insights into the optimization of student flow in
educational institutions. Both methods possess distinct advantages, with DES excelling in simulating
specific events like student arrivals and document processing, while CS offers a more comprehensive
understanding of resource allocation and service demand over time. By combining these models, CSPC
can achieve a more accurate and efficient student flow management system, improving the operational
efficiency of the Registrar's Office.

The comparative analysis of DES and CS benefits from these new references by grounding theoretical
discussions in recent, peer-reviewed research. Sevin et al. [2025] provided a framework to understand
queued dynamics in both discrete and continuous contexts, while Biazen et al. [2025] stressed the
importance of model verification—a critical step when institutions choose between or integrate DES
and CS approaches. Pablo and Ahmad [2021] further demonstrate the tangible benefits of
stochastic simulation in optimizing operational throughput, which is directly related to student service scenarios. Incorporating these perspectives not only strengthens the methodological rigor of
simulation-based optimization but also highlights the practical implications of hybrid and validated
models for educational administration. This integrated approach ensures that both micro-level operational details and macro-level institutional trends are addressed, paving the way for more robust and adaptable student flow management systems.

\subsection{Evaluation of Algorithm}

Analyzing the performance of the Discrete-Event Simulation (DES) and Continuous Simulation (CS) algorithms is essential to better understand student flow in the CSPC Registrar's Office. The evaluation process requires the construction of simulation representations, which effectively reflect registrar
behavior and practice. For example, the modeling of arrival-timed distributions, queue length, and service time enables accurate simulation to mimic an actual registration experience. It has been established in previous studies, including Sevin et al. [2025], that strong queue modeling frameworks
within simulation are the most effective representation of students' interactions within a system.

Furthermore, algorithm performance directly affects efficiency and persistence. Key performance
indicators—such as waited times, queue length, service rate, and throughput—play a crucial role in
evaluating each algorithm's effectiveness in identifying bottlenecks and guiding process improvements.
Comparative evaluation involves both verifying and validating simulation outputs to demonstrate their
reliability and realism. Biazen et al. [2025] emphasize that model verification and validation are
essential components—particularly in dynamic systems like educational organizations—as they ensure
that simulation outputs accurately replicate the behavior of the actual operational system. For DES,
this entails confirming that discrete events, such as student check-ins and document processing, yield
reliable results. For CS, it involves validating the continuous flow of interactions between students and
resource usage over an operational time horizon. When comparing their predictive validity,
computational efficiency, and adaptability to various operational contexts, the strengths of DES in
modeling discrete event processes and the advantages of CS in capturing long-term trends are carefully
balanced.

Finally, the alignment of simulation algorithms with visualization tools such as 3D-based prototypes aids the evaluation process by making complex data less daunting and more accessible to decision-makers. \citeauthor{delarue2021optimizing} [\citeyear{delarue2021optimizing}] and the more recent systematic review by \citeauthor{simio2025differences} [\citeyear{simio2025differences}] confirm that immersive visualization helps support algorithm evaluation, as it draws stakeholders into experimental operational scenarios. Ultimately CSPC administrators will be able to compare operations using DES and CS models within a visual, interactive environment. They can then explore the impact of different flows and the flow optimization strategies in a way that encouraged an intuitive understanding without the need for real-world testing. This level of evaluation provided assurance that not only does the chosen algorithm work on paper, but it also translates into actual improvements in the Registrar’s Office.

\subsection{Synthesis of State-of-the-Art}
Optimizing student flow in educational institutions, particularly within registrar’s offices, had become increasingly vital as student populations grew and administrative complexity rises. Efficient student flow ensured reduced wait time, balanced workloads for staff, and an overall improvement in student satisfaction. Recent literature emphasizes the need for data-driven and simulation-based strategies to address these challenges, especially inhigh-traffic environments where manual processes often became bottlenecks. Studies such as those by Sevin et al. [2025] and Falolo et al. highlight the direct impact of robust queued modeling and digitized registration systems on institutional efficiency and responsiveness. These findings underscore that student flow optimization was not merely a technological upgrade but an institutional transformation that leverages modeling tools and data analytics to simulate and test new policies before full-scale implementation.

Flow simulation stands out as a powerful tool in this context, offering both granular and holistic perspectives on administrative processes. Discrete-Event Simulation (DES) models systems as sequences of distinct events, making it particularly effective for analyzing queued systems, resource constraints, and operational bottlenecks in registrar offices. By simulating individual student arrivals, service times, and document processing, DES provides actionable insights that could drive significant improvements in throughput and wait times, as demonstrated by Martinez & Ahmad [2021] and Marsudi [2020]. Conversely, Continuous Simulation (CS) captured the continuous evolution of the system state over time, making it well-suited for strategic planning and long-term forecasting. CS models could reveal trends in enrollment, resource utilization, and even student well-being, offering administrators a macro-level understanding that complements the event-driven detail of DES.

Comparative studies revealed that while DES was ideal for micro-level, event-driven analysis, CS excels at modeling aggregate flows and long-term institutional trends. The literature, which included systematic reviews and practical case studies, consistently found that each approach had its strengths: DES is preferred for detailed, operational improvements, while CS is invaluable for high-level planning and policy evaluation. Combining these simulation methods with advanced visualization tools like 3D prototypes and extended reality made decision-making even better. Such visualizations made complex data accessible to stakeholders, facilitating the planning and communication of process improvements in a low-risk, interactive environment. Ultimately, the synthesis of these approaches pointed to the value of hybrid, validated simulation models in educational administration. By rigorously evaluating and combining DES and CS, institutions like CSPC can achieve both immediate operational gains and informed, strategic development. The literature stresses the importance of model verification and validation to ensure reliability, as well as the need for stakeholder engagement in the adoption of new systems. This integrated, simulation-based optimization not only addresses current administrative challenges but also equips educational institutions to adapt to future demands, ensuring sustainable improvements in both efficiency and student experience.

\subsection{Gap Bridged by the Study}

Even with the known advantages of both Discrete-Event and Continuous Simulation in the educational administration sector, there remains a significant absence of systematic comparison or side-by-side comparison between the two approaches—specifically in the case of student flow within registrar's offices. Most of the studies examined looked solely at either DES or CS, without comparison to the strengths, weaknesses, and applications of the two types of simulation in different operational conditions. Additionally, there was limited research on using simulation output with 3D visualization tools for real-time, informed decision-making, and this was particularly true in the academic context, where engaged stakeholders and maintaining spatial interpretation of process redesign are important considerations.

This study addresses the gap noted above through detailed comparative analysis of the Discrete-Event Simulation and Continuous Simulation models that are developed for the CSPC Registrar's Office. By engaging both simulation methods and evaluating their effectiveness in important operational
measures, this research offers actionable, evidence-based recommendations on the optimal modeling method to improve student flow. Further, the fact that an interactive 3D-based prototype system has been developed offers an easy-to-use visual reference of the simulation data that administrators can
use to engage in testing the simulation modeling data, adjust variables, discuss process improvements, and communicate the findings of the process before it is attempted in practice. Altogether, this combination of research through the lens of an integrated method contributes to our academic understanding of simulation-based optimization in higher education settings.

%=======================================================%
%%%%% Do not delete this part %%%%%%
\clearpage

\printbibliography[heading=subbibintoc, title={\texorpdfstring{\centering}{} Notes}]
\end{refsection}
    \chapter{Methodology}
\begin{refsection}
This chapter described the step by step processes used in analyzing, modeling, and simulating the flow of students in the Registrars’ Office of Camarines Sur Polytechnic Colleges (CSPC) using both discrete event simulation and continuous simulation models. The study strives to determine the best approaches to reduced traffic, wait times, and service delayed time based on modern simulation techniques and performance evaluation methodologies. The methodology combines a well-organized exploratory research design, careful data collection, appropriate simulation instruments, and a depiction of the underlying reasoning of the study.

\section{Research Design}
This study adopts a constructive research design to optimize student flow in the CSPC Registrar’s Office by developing and evaluating simulation-based solutions. The research focuses on constructing two simulation models, Discrete-Event Simulation (DES) and Continuous Simulation (CS), that served as tools for addressing inefficiencies in student service flow. These models was built using real-world operational data such as student arrival rates, service duration, and queue lengths. The process involves analyzing the current workflows, identifying bottlenecks, and developing simulation-based prototypes to test and improve system performance. A 3D prototype was constructed to visualize and compare both simulation models under the same conditions. Performance metrics such as wait time, queue length, server utilization, and throughput are used to evaluate the effectiveness of each model. The goal was to construct a functional, decision-support tool that enhanced service delivery and informed process improvements in the Registrar’s Office.

\section{Theorems, Algorithms, and Mathematical Models}
This chapter was describe the instrument, procedure/process, and
statistical test that are relevant to the research.

\section{Discrete-Event and Continuous Simulation Flow}
A Simulation Workflow Model was a structured framework used to optimize system processes through simulation techniques. In this study, it supported both the Discrete-Event Simulation (DES) and Continuous Simulation (CS) methods. The model operates by receiving real-world input data (such as student arrival rates and service duration), selecting the appropriate simulation type, and generating performance metrics \cite{chaka2020modeling}. Each component of the flow from input handled to output visualization—designed to mimicked real-world behavior, where simulation nodes represent decision points, pathways reflect student movement or system changes, and final outputs deliver analytical insights for operational improvement. This section presents the structured workflow presented in \ref{fig:firstFig}, the Prototype Flowchart. The process demonstrates a sequence of stages that guide the system from data input to decision-making, as described below:

\begin{figure}[ht]
    \centering
    \includegraphics[width=1\linewidth]{figures/prototype_flowchart.png}
    \caption{Prototype Flowchart}
    \label{fig:firstFig}
\end{figure}

\textit{Start}. The process began at the Start node, which served as the entry point of the prototype system. At this stage, the primary objective was to initiate the workflow and define the parameters of the simulation studied. The starting point ensured that the process followed a logical sequence and allowed the model to progress smoothly from data input to final decision-making.

\textit{Input Data}. After the start, the process continued with the input of real-world data. This included important details such as student arrival patterns, the number of service counters in operation, and the duration of services provided. These inputs were collected directly from Registrar’s Office operations to ensure that the model reflected actual conditions.

\textit{Simulation Model}. Following data input, the system reached a decision point where the type of simulation model was selected. The user must choose between Discrete-Event Simulation (DES) and Continuous Simulation (CS), depending on the level of analysis required.

\textit{DES}. Choosing DES meant the model would simulate students as individual entities. Each arrival, queuing experience, and service transaction is represented in detail, allowing the model to mirror real-world operations closely. This form of simulation highlights the sequence of actions and the weighted experiences of students.

\textit{CS}. Selecting CS shifted the focus from individual behavior to overall system flow. In this approach, the Registrar’s Office is modeled as a continuous process, emphasizing service rates, congestion patterns, and aggregate performance over time.

\textit{Generate Results}. Once the simulation is completed, the system generates a set of performance metrics that describe the efficiency of the Registrar’s Office. This included average waited time, queue length, throughput, and service utilization.

\textit{Render 3D Visualization}. The numerical results of the simulation were then transformed into a 3D prototype visualization. This representation maps out the Registrar’s Office layout and illustrates student flow, queue formations, and areas where congestion typically occurred.

\textit{Output Decision Panel}. The final stage of the prototype flowchart leads to the Output Decision Panel, where the results of the simulation are organized and presented. This panel served as the point where performance metrics, such as waiting time, queue length, throughput, and service utilization, are interpreted and analyzed. The purpose is to consolidate findings from either the Discrete-Event Simulation (DES) or Continuous Simulation (CS) models.

\textit{End}. The process concludes at the End node, where the outputs of the simulation are summarized and presented. At this stage, the comparative results from both DES and CS are highlighted, providing a balanced view of micro-level and macro-level system performance.

\ref{fig:firstFig} illustrated the structured process of student flow optimization, start with the input of real-world data. It proceeds with the selection of a simulation model, either Discrete Event Simulation (DES) for modeling individual events or Continuous Simulation (CS) for representing flow trends. The chosen model then generates performance metrics, followed by 3D visualization of student movement and countered interactions, leading to the output decision panel for performance comparisons and insights.

\subsection{Materials and Statistical Tools}
This chapter was describe the instrument, procedure/process, and statistical test that are relevant to the research.

\subsection{Instrument}
The primary instruments used in this study included observation checklists and data recording sheets designed to capture essential real-world data such as student arrival times, queue lengths, service start and end times, and the number of active service counters. These instruments ensured systematic and accurate collection of operational details needed for simulation input. For simulation development, software tools such as Arena, AnyLogic, or Python libraries like SimPy are employed to model the student flow using Discrete Event and Continuous Simulation techniques. Additionally, 3D visualization was also implemented using React Three Fiber (R3F). It was a React renderer for Three.js, a popular JavaScript library for creating and displaying 3D graphics in a web browser using WebGL. This used to create dynamic representations of the Registrar’s Office layout and student movement patterns.

\subsection{Data Set}
The dataset compiled for this studied included the followed features:

\begin{itemize}
    \item \textbf{Date:}The specific day on which data were collected, representing the actual schedule of student transactions at the Registrar’s Office.
    \item \textbf{Purpose:} The type of service requested by the student, such as Transcript of Records (TOR), Diploma, Certificate of Authentication and Verification (CAV), or other academic documents.
    \item \textbf{Student Arrival Time:} Timestamp indicating when each student arrived at the Registrar’s Office.
    \item \textbf{Service Start and End Time:} The time a student began and finished their transaction at a counter.
    \item \textbf{Service Duration:} The time taken to served each student (calculated as ended time minus started time).
    \item \textbf{Queue Length:} Number of students in line at different time intervals.
    \item \textbf{Number of Active Counters:} The number of service windows opened during different hours of the day.
    \item \textbf{Daily Arrival Rate:} The total number of students arrived within specific time blocks (e.g., hourly).
    \item \textbf{Office Layout Dimensions:} Physical measurements of the office used for 3D visualization (e.g., counter locations, waited area size). 
\end{itemize}

This data collected through on-site observations over multiple days, with additional reference to system logs where available. It served as input for both the Discrete-Event Simulation (DES) and Continuous Simulation (CS) models and was also used to validate the simulation outputs against actual performance metrics. \ref{tbl:sampleTbl1} the Sample Dataset, outlined the key variables and records gathered from the Registrar’s Office.


\begin{table}[!h]
\centering
\caption{Sample Data Set (Registrar’s Office Student Flow)}
\label{tbl:sampleTbl1}
\resizebox{\textwidth}{!}{%
\begin{tabular}{c|c|c|c|c|c|c|c|c}
\hline
\textbf{Date} & \textbf{Student ID} & \textbf{Arrival Time} & 
\textbf{Service Start} & \textbf{Service End} & 
\textbf{Service Duration(min)} & \textbf{Purpose} & \textbf{Queue Length} & \textbf{Counter ID}\\ \hline
04/10/25 & S1  & 6:05 am & 7:55 am & 7:58 am & 3 & TOR            & 18 & C2 \\
04/10/25 & S2  & 6:12 am & 7:58 am & 8:01 am & 3 & Diploma        & 17 & C1 \\
04/10/25 & S3  & 6:20 am & 8:01 am & 8:04 am & 3 & CAV            & 16 & C2 \\
04/10/25 & S4  & 6:27 am & 8:04 am & 8:07 am & 3 & Authentication & 15 & C2 \\
04/10/25 & S5  & 6:35 am & 8:07 am & 8:10 am & 3 & ID             & 14 & C4 \\
04/10/25 & S6  & 6:43 am & 8:10 am & 8:13 am & 3 & TOR            & 13 & C2 \\
04/10/25 & S7  & 6:50 am & 8:13 am & 8:16 am & 3 & COE            & 12 & C3 \\
04/10/25 & S8  & 6:58 am & 8:16 am & 8:19 am & 3 & Authentication & 11 & C1 \\
04/10/25 & S9  & 7:06 am & 8:19 am & 8:22 am & 3 & Diploma        & 10 & C2 \\
04/10/25 & S10 & 7:12 am & 8:22 am & 8:25 am & 3 & ID             & 9  & C4 \\
04/10/25 & S11 & 7:20 am & 8:25 am & 8:28 am & 3 & Authentication & 8  & C2 \\
04/10/25 & S12 & 7:29 am & 8:28 am & 8:31 am & 3 & COE            & 7  & C3 \\
04/10/25 & S13 & 7:37 am & 8:31 am & 8:34 am & 3 & TOR            & 6  & C1 \\
04/10/25 & S14 & 7:45 am & 8:34 am & 8:37 am & 3 & Diploma        & 5  & C1 \\
04/10/25 & S15 & 7:52 am & 8:37 am & 8:40 am & 3 & CAV            & 4  & C2 \\
04/10/25 & S16 & 8:00 am & 8:40 am & 8:43 am & 3 & ID             & 3  & C4 \\
04/10/25 & S17 & 8:09 am & 8:43 am & 8:46 am & 3 & Authentication & 2  & C2 \\
04/10/25& S18 & 8:17 am & 8:46 am & 8:49 am & 3 & Diploma         & 1  & C1 \\
04/10/25 & S19 & 8:24 am & 8:49 am & 8:52 am & 3 & COE            & 0  & C3 \\
04/10/25 & S20 & 8:31 am & 8:52 am & 8:55 am & 3 & TOR            & 0  & C2 \\

-- & -- & -- & -- & -- & -- & -- & -- & -- \\ \hline
\end{tabular}%
}
\end{table}

\ref{tbl:sampleTbl1} presents the sample dataset collected from the Registrar’s Office, which served as the primary input for the Discrete-Event Simulation (DES) and Continuous Simulation (CS) models. The data were gathered through manual observation conducted by the researchers during peak service hours to record the actual sequence of student arrivals, queue lengths, service start and end times, and transaction types. This hands-on data collection ensured that the dataset accurately reflected the real operational flow of the office. The observed data represent the transactions of twenty (20) students recorded on April 10, 2025, across multiple service counters within the Registrar’s Office. Each row corresponds to an individual student transaction and contains key operational variables that describe the flow of students from arrival to service completion. The Date column indicates when the observation took place, ensuring time-based consistency in data collection. The Student ID (S1-S20) uniquely identifies each student, while the Arrival Time records when the student entered the queue. The Service Start and Service End columns specify the exact times when the transaction began and ended, which were used to calculate the Service Duration—consistently measured at three (3) minutes for all transactions, reflecting a uniform service process at the counter. The Purpose column categorizes the type of service availed, including Transcript of Records (TOR), Diploma, Certificate of Authentication and Verification (CAV) or, Authentication, Certificate of Enrollment (COE) and Identification Card (ID), which helps distinguish the variation in service demand. The Queue Length represents the number of students waiting before each service began, starting from eighteen (18) students at the beginning of operations and gradually decreasing to zero (0) as transactions were completed, indicating an orderly and efficient flow of service. The Counter ID indicates the specific service window where each transaction occurred, such as C1, C2, C3, and C4—the CSPC Registrar has four counters, each with an assigned purpose serving the students' needs—corresponding to different operational sections of the Registrar’s Office. For instance, C1 may handle diploma or record requests, C2 is designated for releasing documents, C3 manages certificate of enrollment verification, and C4 assists in identification and authentication processes. These counter assignments reflect their respective service functions, and each counter plays an equally vital role in maintaining efficient office operations. Together, the counters contribute to reducing queue lengths, minimizing waiting times, and ensuring balanced workflow distribution. This dataset—collected through direct manual observation—provides a detailed and realistic representation of student movement and service flow, capturing essential parameters such as arrival patterns, waiting times, and queue dynamics, which serve as the foundation for simulation modeling and performance evaluation in optimizing student service delivery within the Registrar’s Office.


\subsection{Evaluation Methods}

\textit{Average Waiting Time (AWT)}
represents the mean time each student spends waiting before being served. It is calculated by subtracting the arrival time (T ai) from the service start time (T si) for each student, summing the differences for all students, and dividing by the total number of students (n). This metric helps identify how long students typically queue and is critical for assessing service efficiency.

The formula is:

\begin{equation}
    \mathrm{AWT}=\frac{\sum_{i=1}^{n}\left(T_{s i}-T_{a i}\right)}{n}
\end{equation}

\textbf{Where:}
\begin{itemize}
    \item T si =  Service start time of student i
    \item T ai = Arrival time of student i
    \item n = Total number of students
\end{itemize}

\textit{Average Service Time (AST)}
measured how long, on average, a student was served at the counter. It was obtained by subtracting the service started time (T si) from the service ended time (T ei) for each transaction, then averaging the results across all students. This value indicates how quickly staff could complete each student’s transaction, which directly impacts throughput and service quality.

The formula is:

\begin{equation}
    \mathrm{AST}=\frac{\sum_{i=1}^{n}\left(T_{e i}-T_{s i}\right)}{n}
\end{equation}

\textbf{Where:}

\begin{itemize}
    \item T ei = Service end time
    \item T si = Service start time
    \item n = Total number of student
\end{itemize}

\textit{Server Utilization (U)}
was a ratio that showed how effectively the available service counters had been used. It is calculated by dividing the total time spent serving students by the product of the total available service time and the number of counters. A high utilization rate suggested efficient use of resources, while a low rate may indicate underutilized or overstaffed resources.

The formula is:

\begin{equation}
    U=\frac{\text { Total Service Time }}{\text { Total Available Time } \times \text { Number of Counters }}
\end{equation}

\textit{Queue Length (QL) Average}
gave the average number of students waiting in line during a specific timed period. It is calculated by summing the queue lengths recorded at regular intervals (Qi) and dividing by the number of intervals (t). This metric reflected congestion and could help identify peak hours and bottlenecks.

The formula is:

\begin{equation}
    \text { Average QL }=\frac{\sum_{i=1}^{t} Q_{i}}{t}
\end{equation}

\begin{itemize}
    \item Qi = Queue length at time interval i
    \item t = Total number of time intervals observed
\end{itemize}

\textit{Throughput (TP)}
referred to the number of students served within a given period. It was founded by dividing the total number of students processed by the total simulation time. This was a key performance indicator that showed the capacity of the system and how many students could have been effectively served over time.

The formula is:

\begin{equation}
    T P=\frac{\text { Number of Students Served }}{\text { Simulation Time }}
\end{equation}

\subsection{Conceptual Framework}
This section present Conceptual Framework is based on the Input–Process–Output (IPO) model, as shown in \ref{fig:secondFig} used to create a simulation-based system designed to enhance student service flow in the Registrar’s Office. The framework combines Discrete-Event Simulation (DES) and Continuous Simulation (CS) in a 3D interactive environment to deliver a thorough assessment of system performance.

\begin{figure}[ht]
    \centering
    \includegraphics[width=12 cm\linewidth]{figures/CONCEPTUAL FRAMEWORK.drawio (1).png}
    \caption{Conceptual Framework}
    \label{fig:secondFig}
\end{figure}

\textit{Input}. The simulation began with the collection of real-world operational data from the Registrar’s Office, including student arrival times, service times per transaction, and the number of available service counters. To capture dynamic variations in the system, scenario-based parameters were also incorporated, such as peak hours, student volume, and staffing levels. These inputs served as the fundamental datasets for modeling both the baseline and alternative configurations of the Registrar’s Office. 

\textit{Process}. The processing stage took place within a 3D Interactive Simulation Environment, which functions as the central platform for testing and visualizing scenarios. The simulation integrates two modeling approaches: 

\textit{Discrete-Event Simulation (DES)}. This model represents the system through a sequence of discrete events, such as student arrivals, waiting, and service completions. It employed mathematical modeling to monitor variables such as queued length, average waited time, system utilization, and throughput. DES provided micro-level insights into how individual transactions affect overall system performance.

\textit{Continuous Simulation (CS)}. In contrast, CS models the Registrar’s Office as a continuous flow of students. It emphasizes the overall rate of movement and system behavior over time, focusing on macro-level trends rather than individual interactions. This allowed researchers to analyze long-term system dynamics and aggregate patterns.

\textit{3D Environment Incorporates}.  A scenario simulation panel that allows users to manipulate variables in real time, enabling dynamic testing of service configurations and operational adjustments.

\textit{Output}. The outputs of the system included both quantitative and visual performance measurements. Key Performance Indicators (KPIs) consisted of average waited time, queue length, service utilization, throughput, and overall system efficiency. Beyond numerical data, the framework generates visual outputs, such as graphs of queue trends over time and direct comparisons between DES and CS. These outputs provided actionable insights for identified inefficiencies, tested potential interventions, and enhanced decision-making.

\textit{Feedback}. An integral component of the framework was its feedback loop. Insights obtained from simulation results cycled backend into the input stage, enabling iterative refinement of the system. For example, if results revealed high congestion during peak hours, administrators could adjust staffing levels or increase service counters and rerun the simulation. This continuous feedback mechanism ensured adaptability and fostered evidence-based decision-making in service operations. Systematically combining real-world data, scenario-based modeling, dual simulation techniques, measurable outputs, and iterative feedback, this framework provided a robust decision-support tool for student service management. It enhanced the Registrar’s ability to anticipate operational bottlenecks and allocate resources effectively and improved overall service delivery efficiency \cite{Nickson2021Translational}.  
 

%=======================================================%
%%%%% Do not delete this part %%%%%%
\clearpage

\printbibliography[heading=subbibintoc, title={\texorpdfstring{\centering}{} Notes}]
\end{refsection}
    \chapter{Results and Discussion}
This chapter presented the results and analysis of the study, organized according to the researched objectives. Each section directly addressed a researched objective, demonstrating how the Discrete-Event Simulation (DES) and Continuous Simulation (CS) models were implemented, evaluated, and visualized. The organization ensured that findings were systematically connected to the study’s goals.

\section{4.1 Implementation of Discrete-Event and Continuous Simulation Models}
The implementation of the Discrete-Event Simulation (DES) and Continuous Simulation (CS) models was carried out using dedicated scientific computing tools built in Python: the SimPy 4.0.1 framework for DES and SciPy’s solve\_ivp method (with the RK45 solver)} for CS.

For the DES model, the simulation was implemented using SimPy 4.0.1, a process-based discrete-event simulation library built on Python’s generator architecture, which allows asynchronous event scheduling and resource management. The study began by collecting and preparing data from the Registrar’s Office, such as student arrival times, service durations, and queue lengths, which were then standardized to ensure reliability. Using the simpy.Environment() class, an event-driven simulation environment was created, enabling continuous tracking of individual student processes. Service counters were defined as shared resources using simpy.Resource(), limiting concurrent service capacity and modeling congestion in high-demand scenarios. Each simulated “student” entity followed a lifecycle process encompassing arrival, waiting, service, and departure. The time between these states was managed using env.timeout(), which advanced the simulation’s event clock without iterating through every unit of real time. Performance metrics, including waiting time, queue length, throughput, and resource utilization, were monitored throughout the simulation. The resulting model effectively captured queue dynamics and congestion variability consistent with real-world operations. Finally, verification and validation were conducted by comparing simulation results with observed registrar data, ensuring that the DES accurately represented operational conditions.

For the Continuous Simulation (CS) model, the SciPy framework was employed to represent student flow behavior using mathematical modeling rather than discrete event tracking. In particular, the  solve\_ivp function with the Runge-Kutta (RK45) integration method was utilized to solve a system of differential equations describing the change in the number of students within the system over time. 
The model continuously tracked the states of waiting, served, and completed students with time-based changes represented through numerical solutions rather than discrete state jumps. This provided smooth approximations of system dynamics and made it possible to observe overall flow stability and transition trends. Simulated trends revealed efficiency metrics such as an approximately 12-minute average waiting time, a throughput rate of around 12.6 students per hour, and a service utilization rate close to 49.5\%. Validation and sensitivity analysis were later performed to ensure robustness under varying input conditions, confirming that the differential equation model accurately captured the temporal patterns of student flow. These results highlight the suitability of the Continuous Simulation approach for examining long-term system behavior and average performance trends rather than individual service interactions. Consequently, the CS model served as a complementary analytical tool to the Discrete-Event Simulation model, supporting strategic evaluation and capacity planning for the Registrar’s Office.


\subsubsection{4.1.1 Discrete-Event Simulation (DES) Development}
The Discrete-Event Simulation (DES) model was implemented to represent and analyze the detailed movement of students within the Registrar’s Office. The model simulated how students arrived, waited, and were served at different counters based on real-world operational data, allowing the system to capture realistic service interactions, queue formations, and variations in workload throughout the operational period.
\ref{tbl:sampleTbl2} presents the Simulation Parameters for DES Model.

\begin{table}[!h]
\centering
\caption{Simulation
\ Parameters for DES Model}
\label{tbl:sampleTbl2}
\begin{tabular}{l|c}
\hline
\textbf{Parameter} & \textbf{Value/Description} \\ \hline
Number of Service Counters        & 4  \\
Average Service Time (minutes)      & 9.33 \\
Average Time Between Arrivals (minutes) & 2  \\
Simulation Duration (minutes)           & 480 (8 hours)  \\
Simulation horizon        & 480 minutes (8 hours)  \\
Maximum Number of Students	&353 \\
Initial Students in Queue	&353 \\
Expected Load	&240 students \\
Queue Discipline	&First-Come, First-Served (FCFS) \\
Simulation Type	&Discrete-Event Simulation (DES) \\ \hline
\end{tabular}
\end{table}

The DES model was developed to capture the step-by-step transactions of students in the Registrar’s Office, treating each process—such as arrival, waiting, service start, and service completion—as a discrete event that changes the system’s state. In the simulation, students are modeled as entities, while service counters act as finite resources, reflecting the real-world service environment. The simulation follows a First-Come, First-Served (FCFS) policy, ensuring that students are served in the order they arrive, consistent with the actual operations of the Registrar’s Office. This setup allows the model to replicate realistic system behaviors such as queue build-ups during peak periods and service recovery after delays.

Service times are derived from empirical data, averaging 9.33 minutes per transaction, while arrival intervals average 2 minutes, indicating frequent student arrivals and potential queue formation. The model runs for 480 minutes (8 hours), representing a full workday, and continuously monitors metrics such as average waiting time, queue length, throughput, and service utilization. By simulating student transactions at the individual level, the DES model provides a highly detailed and dynamic view of operational performance, enabling the identification of bottlenecks, idle time, and peak-hour congestion. This event-driven approach ensures a realistic depiction of student flow and serves as a foundation for evaluating efficiency improvements and optimization strategies.

\subsubsection{4.1.2 Continuous Simulation (CS) Development}
The Continuous Simulation (CS) model was developed to provide an aggregate and system-wide perspective of student flow in the Registrar’s Office. Unlike the Discrete-Event Simulation (DES), which tracks individual students as separate entities, the CS approach models the system in terms of flows, rates, and state variables that evolve continuously over time. The main state variable in the model is the number of students in the system, N(t), which changes as students arrive, join the queue, and complete service.

The governing equation of the system is expressed as:

\[
\frac{dN}{dt} = \lambda(t) - \mu(t) \cdot S(t)
\]


where $\lambda(t)$ represents the arrival rate of students, $\mu(t)$ is the service rate, and $S(t)$ is the number of active service counters. In this study, arrival and service rates were estimated from the preprocessed dataset and aggregated into smoothed hourly averages to approximate the continuous inflow and outflow of students. This mathematical structure enables the CS model to approximate overall congestion trends and long-term flow stability, providing a broader understanding of system behavior.

Following this system-level formulation, the Continuous Simulation (CS) model was configured using parameters that reflect actual operating conditions in the Registrar’s Office. These parameters translate the abstract mathematical structure into a realistic representation of daily operations by defining the simulation duration, service capacity, arrival intensity, and student population. Through this configuration, the CS model bridges theoretical flow dynamics with practical operational settings, allowing the continuous behavior of student flow to be observed over a full working day. 
 \ref{tbl:sampleTbl3} presents the Simulation Parameters for CS Model.

\begin{table}[!h]
\centering
\caption{Simulation Parameters for CS Model}
\label{tbl:sampleTbl3}
\begin{tabular}{l|c}
\hline
\textbf{Parameter} & \textbf{Value/Description} \\ \hline
Number of Service Counters	&4 \\
Average Service Time (minutes)	&9.33 \\
Average Time Between Arrivals (minutes)	&2 \\
Simulation Time (minutes)	&480 (8 hours) \\
Maximum Number of Students	&353 \\
Initial Students in Queue	&353 \\
Expected Load	&240 students \\
Simulation Type	&Continuous Simulation (CS) \\ \hline
\end{tabular}
\end{table}

\ref{tbl:sampleTbl3} the simulation horizon was set to 480 minutes (8 hours), replicating a full workday operation in the Registrar’s Office. Within this timeframe, four active service counters were configured to represent the available service points, ensuring that the model could capture multi-counter operations under real-world conditions. The average service time was set at 9.33 minutes per transaction, based on empirical data, while the average time between arrivals was 2 minutes, reflecting a steady and frequent inflow of students. In total, 353 students were included in the simulation, serving both as the maximum population and as the initial queue load, accurately representing peak operational hours where congestion typically occurs. These parameters allowed the Continuous Simulation (CS) model to approximate the dynamic flow of students over time, emphasizing overall system behavior rather than individual event tracking. The Discrete-Event Simulation (DES), which models each transaction step-by-step, the CS model focuses on continuous system evolution, showing how the number of students changes throughout the operational period. This makes it particularly effective in understanding macro-level performance, such as long-term utilization trends, system capacity, and average congestion patterns. It enables administrators to predict how the office would perform under varying workloads or staffing conditions, offering valuable insights for long-term planning and scheduling.

A significant strength of the CS approach is its ability to present a smooth and stable visualization of system behavior, ideal for strategic decision-making and forecasting. However, it lacks sensitivity to short-term fluctuations—such as sudden surges in arrivals, short service delays, or temporary bottlenecks—which are better captured by the DES model. Despite this limitation, the CS and DES models complement each other: DES provides micro-level precision for daily operations, while CS delivers a broader, aggregated view of system efficiency. Combined, they form a powerful analytical framework for optimizing registrar operations through both tactical and strategic perspectives.


\subsubsection{4.1.3 Initial Results and Visualizations of DES and CS}
To better understand the operational dynamics of the Registrar’s Office, the initial results of the Discrete-Event Simulation (DES) and Continuous Simulation (CS) models were compared using two key performance indicators: average waiting time over time and average queue length over time. These visualizations highlight how each modeling approach captured system performance within the same operating horizon and revealed differences in sensitivity and accuracy.
These performance indicators were selected because they directly reflect student experience and service efficiency within the Registrar’s Office. Average waiting time over time illustrates how quickly students are processed under varying demand levels, while average queue length over time shows how congestion builds up and dissipates throughout the operating period. Examining these indicators simultaneously allows for a clearer comparison of how DES and CS respond to changes in arrival patterns, service capacity, and workload intensity. These comparisons also help identify periods where operational strain is most significant and where resources may be underutilized. Such insights support a more informed evaluation of how each simulation model represents real-world registrar operations under varying demand conditions. As a result, the visual outputs provide meaningful insights into the strengths and limitations of each modeling approach when applied to real operational conditions. \ref{fig:threeFig}, displayed the Comparison of Average Waiting Time.

\begin{figure}[H]
    \centering
    \includegraphics[width=10cm\linewidth]{figures/figure4.png}
    \caption{Comparison of Average Waiting Time}
    \label{fig:threeFig}
\end{figure}

\ref{fig:threeFig} showcased the comparison of average waiting times between the Discrete-Event Simulation (DES) and Continuous Simulation (CS) models across the Registrar’s Office operations. The DES model, represented by the blue line, demonstrated fluctuating waiting times throughout the simulation period, accurately reflecting real-world variations in student arrivals and service capacity. During the morning hours, the average waiting time remained high at around 40 to 45 minutes due to the early buildup of students in the queue and limited service counters. Around midday, waiting times dropped significantly to about 20 minutes, indicating a temporary reduction in congestion before increasing again in the afternoon as new students arrived.

The CS model, shown in red, maintained a nearly constant average waiting time of approximately 12 minutes throughout the entire 8-hour period. This steady pattern occurs because the CS model represents the system as a continuous flow of entities, averaging arrival and service rates rather than capturing individual events. While this approach provides a smoother and more stable representation of the system, it lacks sensitivity to real-time fluctuations—such as sudden increases in arrivals or temporary service delays—that are common in actual registrar operations. 

The DES model proved to be more effective in capturing real-time dynamics, queue fluctuations, and variations in service performance. It offers a more realistic depiction of operational efficiency and short-term system behavior. On the other hand, the CS model is valuable for identifying long-term patterns and overall process stability. Together, both models complement each other—DES provides detailed, event-level insights for operational improvement, while CS supports high-level planning and strategic decision-making. \ref{fig:fourthFig} displayed the Comparison of Queue Length.

\begin{figure}[H]
    \centering
    \includegraphics[width=10cm]{figures/figure5.png}
    \caption{Comparison of Queue Length}
    \label{fig:fourthFig}
\end{figure}

\ref{fig:fourthFig} presents the comparison of average queue lengths between the Discrete-Event Simulation (DES) and Continuous Simulation (CS) models throughout the Registrar’s Office operations. The DES model, represented by the blue line, showed an average queue length of approximately 20 students, indicating a higher level of congestion during the simulation period. This pattern reflects the discrete and event-driven nature of the DES model, where each student arrival, service initiation, and completion caused variations in the queue length. As the simulation progressed, short-term fluctuations and peaks emerged due to the uneven distribution of arrivals and the limited number of available service counters. These variations mirror actual operational conditions, in which queues tend to build up rapidly during peak hours and decrease when service efficiency improves. 

The CS model, represented by the red line, maintained a nearly constant average queue length of around 15 students. This stability results from the continuous nature of the CS approach, which models the system using averaged rates rather than individual events [\citeyear{Raczynski2020}]. Instead of responding to discrete arrivals and service completions, the CS model captures the overall flow dynamics in a smooth and continuous manner, depicting the equilibrium behavior of the system under consistent arrival and service conditions. Although this model simplifies the representation of queues, it provides valuable insights into long-term operational stability and system capacity. 

The DES model proved to be more sensitive to short-term variations and queue fluctuations, offering a realistic reflection of the dynamic interactions between students and service counters. On the other hand, the CS model is more suitable for high-level analysis and long-term performance evaluation, as it illustrates the general flow stability without focusing on micro-level events. Together, both models complement each other—DES offers detailed insights for operational adjustments, while CS supports strategic planning by emphasizing overall system steadiness.

\subsubsection{4.1.4 DES/CS Performance by Service Type}
The performance comparison of various registrar service purposes-Certificate of Authentication and Verification (CAV), Certificate of Enrollment (COE), Diploma, ID Validation, and Transcript of Records (TOR)—based on four key indicators: average waiting time, average service time, throughput, and students served. It highlights the differences in service efficiency and workload distribution across each transaction type. 
The comparison further reveals how variations in transaction complexity and processing requirements influence overall system performance. Services that require multiple verification steps or additional processing tend to exhibit longer waiting and service times, while more routine transactions demonstrate higher throughput and faster completion rates. These distinctions provide valuable insight into how different registrar services contribute to congestion patterns and resource demand, supporting more balanced staffing allocation and improved scheduling strategies. \ref{fig:fifthFig}, displayed the Discrete-Event and Continuous Simulation Performance by Service Type.

\begin{figure}[H]
	\centering
         % Row 1
	\begin{subfigure}[b]  {0.40\textwidth}
		\centering
		\fbox{\includegraphics[width=1\textwidth]{figures/metrics1.png}}
		\caption{DES Wait Time (minutes) and CS Wait Time (minutes)}
		\label{fig:firstFig:a}
	\end{subfigure}
	\hfill
	\begin{subfigure}[b]{0.40\textwidth}
		\centering
		\fbox{\includegraphics[width=1\textwidth]{figures/metrics2.png}}
		\caption{DES Service Time (minutes) and CS Service Time (minutes)}
		\label{fig:firstFig:b}
	\end{subfigure}
	
	\vspace{1em}
	
	% Row 2
	\begin{subfigure}[b]{0.40\textwidth}
		\centering
		\fbox{\includegraphics[width=1\textwidth]{figures/metrics3.png}}
		\caption{DES Throughput (students/hour) and CS Throughput (students/hour)}
		\label{fig:firstFig:c}
	\end{subfigure}
	\hfill
	\begin{subfigure}[b]{0.40\textwidth}
		\centering
		\fbox{\includegraphics[width=1\textwidth]{figures/metrics4.png}}
		\caption{DES Students Served and CS Students Served}
		\label{fig:firstFig:d}
	\end{subfigure}
	
	\caption{DES/CS Performance by Service Type}
	\label{fig:fifthFig}
\end{figure}

\textit{a.) DES Wait Time (minutes) and CS Wait Time (minutes)}

This chart shows that DES wait times vary widely depending on the type of transaction. For example, TOR and diploma requests have noticeably higher wait times in DES, while services like COE and CAV show lower waiting times. This indicates that DES captures actual service congestion and demand differences across transaction types. The CS model maintains very low and nearly constant wait times for all services. This means CS smooths out real queue fluctuations and does not reflect sudden surges or peak-hour delays, making its results more generalized compared to DES.


\textit{b.) DES Service Time (minutes) and CS Service Time (minutes)}

The DES service time values vary across different purposes, showing that some services (e.g., Diploma) require more processing time, while others (e.g., Authentication, ID Validation) are completed faster. This reflects actual workload differences among service types.
On the other hand, the CS model shows almost the same service time for every purpose, which means it uses average service duration rather than distinguishing based on transaction complexity.

\textit{c.) DES Throughput (students/hour) and CS Throughput (students/hour)}

Throughput results indicate that DES can process more students per hour for certain services, such as COE and TOR, because it accounts for situations where counters work faster or queues clear quickly. In comparison, CS throughput is lower and more uniform, showing that it assumes a steady, averaged service rate. This makes CS less responsive to real-time speed variations in the workflow.

\textit{d.) DES Students Served and CS Students Served}

The DES model shows high variation in how many students are served for each transaction type, with higher counts for faster transactions like Authentication and TOR and lower counts for slower ones like Diploma. The CS model, however, shows no variation in students served per purpose because it does not track individual transactions. Instead, it only estimates the general system flow, meaning it cannot represent which services handle more or fewer students during the day.
    
 
\subsubsection{4.1.5 DataSet Collection}
The dataset used as input for both the Discrete-Event and Continuous Simulation models. It contains key operational variables such as arrival times, queue lengths, service durations, and counter assignments. These variables were extracted from the collected observational data and served as the basis for understanding system behavior. As shown in the \ref{tbl:sampleTbl7}, patterns such as early congestion before counter operations and uneven distribution of workloads across service counters were identified.


\begin{table}[!h]
\centering
\caption{Registrar’s Office Student Flow Dataset}
\label{tbl:sampleTbl7}
\resizebox{\textwidth}{!}{%
\begin{tabular}{c|c|c|c|c|c|c|c|c}
\hline
\textbf{Date} & \textbf{Student ID} & \textbf{Arrival Time} & 
\textbf{Service Start} & \textbf{Service End} & 
\textbf{Service Duration(min)} & \textbf{Purpose} & \textbf{Queue Length} & \textbf{Counter ID}\\ \hline
10/13/25 & S1  & 6:07 am  & 8:00 am   & 8:03 am  & 3 & TOR            & 15 & C1 \\
10/13/25 & S2  & 6:10 am  & 8:03 am   & 8:06 am  & 3 & Diploma        & 14 & C1 \\
10/13/25 & S3  & 6:25 am  & 8:06 am   & 8:09 am  & 3 & CAV            & 13 & C1 \\
10/13/25 & S4  & 6:32 am  & 8:09 am   & 8:12 am  & 3 & Authentication & 12 & C1 \\
10/13/25 & S5  & 6:39 am  & 8:12 am   & 8:15am   & 3 & TOR            & 11 & C1 \\
10/13/25 & S1  & 8:40 am  & 10:30 am  & 10:32 am & 2 & Authentication & 30 & C2\\
10/13/25 & S2  & 8:46 am  & 10:32 am  & 10:35 am & 3 & CAV            & 30 & C2\\
10/13/25 & S3  & 8:51 am  & 10:35 am  & 10:38 am & 3 & Diploma        & 30 & C2\\
10/13/25 & S4  & 8:56 am  & 10:38 am  & 10:41 am & 3 & CAV            & 31 & C2\\
10/13/25 & S5  & 9:01 am  & 10:41 am  & 10:43 am & 2 & TOR            & 31 & C2\\
10/13/25 & S1  & 6:34 am  & 8:00 am  & 8:02 am   & 2 & COE            & 8  & C3\\
10/13/25 & S2  & 6:38 am  & 8:02 am  & 8:04 am   & 2 & COE            & 7  & C3\\
10/13/25 & S3  & 6:46 am  & 8:04 am  & 8:07 am   & 3 & COE            & 6  & C3\\
10/13/25 & S4  & 6:53 am  & 8:07 am  & 8:09 am   & 2 & COE            & 5  & C3\\
10/13/25 & S5  & 7:26 am  & 8:09 am  & 8:11 am   & 2 & COE            & 4  & C3\\
10/13/25 & S1  & 6:37 am  & 8:00 am  & 8:02 am   & 2 & ID VALIDATION  & 8  & C4\\
10/13/25 & S2  & 6:47 am  & 8:02 am  & 8:04 am   & 2 & ID VALIDATION  & 7  & C4\\
10/13/25 & S3  & 6:58 am  & 8:04 am  & 8:05 am   & 1 & ID VALIDATION  & 6  & C4\\
10/13/25 & S4  & 7:16 am  & 8:05 am  & 8:06 am   & 1 & ID VALIDATION  & 5  & C4\\
10/13/25 & S5  & 7:27 am  & 8:06 am  & 8:07 am   & 1 & ID VALIDATION  & 4  & C4\\
-- & -- & -- & -- & -- & -- & -- & -- & -- \\ \hline
\end{tabular}%
}
\end{table}

\ref{tbl:sampleTbl7} presented the operational data needed for the development of both DES and CS models when analyzing the flow of students in the Registrar's Office. These include essential variables such as arrival times, service start and end times, service duration, queue length, transaction purpose, and assigned counter. These variables provide the main parameters for modeling how students enter the system, wait in queues, receive service, and exit the process.

Using this dataset, the DES model can accurately replicate the sequence of events student arrivals, queue formation, and service completion—while the CS model can represent the continuous change in system congestion and service load over time. The table shows early arrival records with the high queue lengths recorded at the very start, representing the peak congestion period even before the official hours of service. At the same time, service durations remain constant at 2–3 minutes, providing a good basis for estimating the processing capacity and finding bottlenecks across counters.

Additionally, the operational profiles support the further refinement and testing of the developed simulation models. By integrating the arrival peaks observed, consistent service durations, and activities specific to each counter, the DES and CS models can simulate how varied system configurations affect the overall students' flow. In this respect, the study is able to simulate several improvement strategies, such as the reallocation of counters, changes in service start times, or workload leveling, in order to identify the interventions that will result in the greatest queue length and waiting time reduction. In this context, the data provides not only input regarding the construction of the models but also a benchmark with which to understand the effects of the suggested operational changes.


\subsubsection{4.1.5 Simulation Platform Flow}
The implementation of both Discrete-Event Simulation (DES) and Continuous Simulation (CS) models, a comprehensive simulation platform architecture was designed and developed. This architecture served as the foundation of the entire system, enabling seamless interaction between data processing, simulation execution, and real-time visualization. It integrates multiple functional components into a unified platform that supported both academic research and practical decision-making. Through the integration of a Flask-based backend for simulation logic with a React.js-based frontend for user interaction, the platform ensured efficient data flow, flexible configuration, and interactive visualization. It was built with scalability, modularity, and user accessibility in mind, allowing stakeholders to configure simulation parameters, upload datasets, run analyses, and visualize results dynamically. \ref{fig:seventhFig} displayed the diagram of the Simulation Platform Flow.


\begin{figure}[H]
    \centering
    \includegraphics[width=1\linewidth]{figures/arch 1.png}
    \caption{Simulation Platform Flow}
    \label{fig:seventhFig}
\end{figure}

\ref{fig:seventhFig}, showed a comprehensive overview of how the integrated system operates, connecting both backend and frontend components into a unified, interactive, and scalable solution. The platform was designed to support the full cycle of simulation activities—from data processing and model execution to real-time visualization and analysis—ensuring seamless communication between components.

The backend, developed using Flask, functions as the core engine that drives the simulation logic. It features a RESTful API serving as the main communication bridge between the backend and frontend. This API interacts with both the Discrete-Event Simulation (DES) and Continuous Simulation (CS) models, which execute simulation computations based on user-defined inputs. Additionally, a PDF Parser module automates the extraction of data from structured documents and stores the processed data in the centralized Data Storage, which manages datasets and model parameters for consistent use across simulations. On the frontend, built with React.js, users can interact with the system through intuitive interfaces such as Simulation Control and Dataset Upload. These interfaces allow users to configure simulation parameters, upload datasets, and initiate simulation runs. Once the backend completes processing, the resulting data is transmitted to the Metrics Dashboard, where it is analyzed and displayed as key performance indicators (KPIs). The data is also visualized in the 3D Visualization module, which dynamically renders the simulated environment, enabling users to clearly observe system behavior and performance outcomes.

The upper layer of the platform are several enhanced features designed to enrich usability and flexibility. These include CSPC Branding for institutional identity, Realistic Timed Mode to simulate actual operational timelines, Fast Analysis Mode for rapid computation and preview, and a Dynamic Counter Layout that allows real-time reconfiguration of the simulation environment. These features collectively improve the platform’s adaptability for different operational needs and ensure a smooth, responsive user experience—from data ingestion and simulation execution to visualization and analysis. The simulation platform flow not only streamlines the execution of both DES and CS models but also bridges data-driven analysis with administrative decision-making. Its modular and interactive design enables iterative testing and optimization of registrar operations, allowing stakeholders to visualize the effects of adjustments in staffing, counter assignments, and service schedules in real time. By integrating backend computation and frontend visualization, the platform effectively supports the study’s objective of optimizing student flow and provides a scalable framework that can be extended to other academic service units within CSPC.



\section{4.2 Evaluation of Model Performance}

For the second objective, the performance of the models was evaluated using average waiting time, queue length, service utilization, throughput, and system efficiency. Both the Discrete-Event Simulation (DES) and Continuous Simulation (CS) models were tested under the same dataset and operational conditions. The following subsections presented the results for each metric, highlighting similarities and differences between the two models. \ref{tbl:sampleTbl5} presented the Simulation Model Performance Comparison.


\begin{table}[!h]
\centering
\caption{Simulation Model Performance Comparison}
\label{tbl:sampleTbl5}
\begin{tabular}{l|c|c|c}
\hline
\textbf{Metric} & \textbf{DES} & \textbf{CS} & \textbf{Difference} \\ \hline
Average Wait Time      & 59.60 min   & 12.00 min  & +147.60 min (-+396.7\%) \\
Throughput             & 21.9/hr     & 12.6/hr      & +9.3/hr (+73.3\%) \\
Resource Utilization   & 94.3\%     & 49.5\%      & +44.8 (+90.7\%) \\
Computation Time       & 21.35 ms   & 64.60 ms   & -43.25 ms (-67.0\%\%) \\
Overall Efficiency     & 34.5\%    & 19.5\%       & +15.0 (+77.1\%) \\ \hline
\end{tabular}
\end{table}

 A clearly understanding of how each simulation model performed required a detailed evaluation of their behavior under identical conditions. The evaluation focused on the five primary performance metrics: average waiting time, queue length, service utilization, throughput, and system efficiency. Each metric offers a distinct perspective on how effectively the models captured real operational dynamics inside the Registrar’s Office. The following sections discuss these results individually, providing insights into the strengths and limitations of both DES and CS.



\begin{itemize}
    \item \textit{Average Waiting Time.} The average waiting time indicates how long students wait from their arrival until the start of service. The DES model recorded a significantly lower waiting time of 59.60 minutes compared to the CS model’s 12.00 minutes. This large gap highlights how DES, which tracks individual student events, captures real-time variations and minimizes waiting delays. In contrast, the CS model relies on continuous flow rates, which tend to average out fluctuations and may underestimate congestion during peak hours.
\end{itemize}

\begin{itemize}
    \item \textit{Average Queue Length.} The average queue length represents the number of students waiting at any given time. Results showed that DES effectively minimized queues and managed variations more dynamically than CS.
\end{itemize}

\begin{itemize}
    \item \textit{Throughput.} Throughput, which measures the number of students served per hour, also favored DES with a rate of 21.9 students per hour, compared to CS’s 12.6. Although the difference seems small, it demonstrates DES’s ability to model event-driven operations and complete more transactions under identical conditions, providing higher responsiveness and operational accuracy.
\end{itemize}

\begin{itemize}
    \item \textit{Service Utilization and System Efficiency.} Resource utilization measures how often service counters are actively engaged in processing students. DES recorded a higher utilization rate of 94.3\%, showing that counters are efficiently used without excessive idle periods. The CS model, at 49.5\%, indicated under utilization due to its generalized representation of the process. DES also outperformed CS in computation time, processing faster with 21.35 milliseconds compared to CS’s 64.60 milliseconds. Finally, in terms of overall system efficiency, DES achieved 34.5\% compared to CS’s 19.5\%, proving its capability to simulate real-time operations more accurately. These results confirm that DES provides a more detailed, efficient, and responsive representation of the Registrar’s Office workflow compared to the broader but less dynamic CS model.
\end{itemize}

\subsubsection{4.2.1 Comparison of DES and CS Model Outcomes Across Performance Metrics}
The graph presented a direct comparison of key performance indicators between Discrete Event Simulation (DES) and Continuous Simulation (CS), emphasizing how each method handled system efficiency. DES (shown in blue) demonstrates a significant advantage over CS (in red) in both weighted time and queued length, with values that were nearly negligible.
\ref{fig:eightFig} displayed the Comparison of DES and CS Model Outcomes Across Performance Metrics.

\begin{figure}[ht]
    \centering
    \includegraphics[width=1\linewidth]{figures/figure 7.png}
    \caption[Comparison of DES and CS Model] {Outcomes Across Performance Metrics}
    \label{fig:eightFig}
\end{figure}

The Side-by-Side Metrics graph provides a visual comparison of four major performance indicators—average waiting time, queue length, throughput, and utilization—between the Discrete-Event Simulation (DES) and Continuous Simulation (CS) models. The blue line (DES) and red line (CS) illustrate how each model responded to operational dynamics during the 480-minute simulation period. The DES model recorded a significantly lower average waiting time, which means students are able to receive services more quickly after arrival. This suggests that DES successfully modeled real-time event handling, queue clearing, and service prioritization, resulting in a more efficient system flow. Conversely, the CS model displayed longer waiting times, representing a more static and generalized view that fails to capture short-term variations in student arrivals and service completions.

In terms of queue length, the DES model again demonstrated superiority, maintaining shorter and more dynamic queue levels throughout the simulation. It effectively reflected actual operational fluctuations, where queues increase during peak hours and decrease as service counters clear transactions. Meanwhile, the CS model showed consistently longer queues due to its aggregated approach, which tends to overestimate congestion since it smooths out real-time data. This contrast highlights DES’s advantage in modeling detailed, event-driven scenarios that are closer to actual office behavior. When evaluating throughput, or the number of students served per hour, both models produced relatively similar results. However, DES had a slight edge, indicating a higher completion rate within the same time frame, confirming its ability to adapt faster to changing demand and service durations.

Resource utilization and overall efficiency also reveal important differences between the two models \citeauthor{Ivo2021simulation} [\citeyear{Ivo2021simulation}]. The CS model exhibited higher utilization values, meaning its resources were active for a larger portion of the simulation time. However, this higher utilization did not translate to better performance—it came with longer waiting times and higher congestion, suggesting inefficiency under variable workloads. In contrast, the DES model achieved balanced utilization and much greater system efficiency, optimizing service time while preventing counter overload. The graph overall demonstrates that DES is the more responsive and accurate simulation technique, capable of capturing both micro-level operational dynamics and real-time service variations. Meanwhile, the CS model, though less precise, remains useful for long-term analysis and planning since it highlights broader system trends and capacity limits. Together, these models offer a comprehensive view of registrar operations—DES for detailed, short-term optimization and CS for strategic, macro-level forecasting.

\subsubsection{4.2.2 Feedback Loop Mechanism in the Simulation Framework}
An essential component of the simulation framework was the feedback loop, which played a critical role in refining the overall system. Once the simulations were executed and their outputs analyzed, the results were systematically feedback into the input stage to improve subsequent iterations of the model. This iterative process allowed for the continuous enhancement of the Registrar’s Office operations by identifying weak points and testing alternative strategies in real time.
For example, if the output indicated that queued lengths and waited times significantly increased during peak hours, this insight fed back into the input parameters by adjusting student arrival rates, service times, or the number of available counters. Similarly, if resource utilization was found to be either too low (indicating idle capacity) or too high (indicating system strain), staffing levels and scheduling strategies were modified to find an optimal balance. This process ensured that each new simulation that ran was based on improved, data-driven parameters rather than static assumptions.
By cycling performance insights back into the model’s inputs, the system enabled evidence-based decision-making and adaptive operational planning. This feedback mechanism not only enhanced the accuracy and relevance of the simulation but also ensured that the 3D interactive environment could support real-world scenarios tested and optimized. Ultimately, the feedback loop transformed the simulation from a one-time analysis tool into a dynamic decision-support system and allowed the Registrar’s Office to continuously refine its operations for improved efficiency and service quality.

\section{4.3 3D Prototype Visualization Development}
 This section presents the 3D prototype designed to visualize the simulation results derived from both the Discrete-Event Simulation (DES) and Continuous Simulation (CS) models. The prototype was developed to serve as an interactive tool that translates simulation data into a dynamic and realistic environment, enabling users to observe and analyze the flow of students, queue behavior, and service counter utilization in the Registrar’s Office.

\subsubsection{4.3.1 Prototype Design and Implementation}
For the last objective, a 3D prototype was developed based on the model outputs of both the Discrete-Event Simulation (DES) and Continuous Simulation (CS) models. The purpose of this prototype was to transform simulation data into an interactive, visual environment that demonstrates the actual flow of students, queue behavior, and counter utilization in the Registrar’s Office. The prototype functioned as a visual decision-support tool, bridging the gap between analytical simulation results and operational insights for administrators. \ref{fig:nineFig}, showcased the 3D Prototype of the Registrar’s Office.

\begin{figure}[H]
    \centering
    \includegraphics[width=1\linewidth]{figures/Screenshot 2025-11-08 091238.png}
    \caption{3D Prototype of the Registrar’s Office}
    \label{fig:nineFig}
\end{figure}

\ref{fig:nineFig} the 3D environment closely replicated the physical layout of the Registrar’s Office, including service counters, waiting areas, entrances and exits, and seating arrangements. Student entities were animated to follow simulated pathways—from arrival to queueing, service, and departure—mirroring the outputs generated by the DES and CS models. Staff stations and counter distributions were also represented to provide a realistic depiction of operational activities. By integrating simulation data directly into the 3D layout, the system allowed users to visualize how arrival rates, service durations, queue lengths, and waiting times influenced the movement and density of students at any point in the simulation.

The 3D model system enables data-driven decision-making because it uses actual operational inputs—such as congestion levels, service delays, arrival patterns, and processing capacity—and converts them into a visible and quantifiable representation of real behavior. Through this integration, every movement and queue buildup displayed in the 3D model corresponds to actual simulation values, ensuring that decisions are grounded on evidence rather than assumptions. This provides administrators with a more reliable understanding of peak-hour congestion, counter performance, and workflow bottlenecks affecting service delivery.

Furthermore, the prototype significantly enhances operational planning by supporting the analysis of “what-if” scenarios without affecting day-to-day registrar operations. Administrators can experiment with alternative staffing arrangements, additional counters, improved layouts, or policy adjustments and immediately compare how these changes affect waiting times, queue lengths, and overall throughput. This makes the 3D model a powerful and practical tool for optimizing workflow strategies before implementation in the real setting.

Overall, the 3D prototype system transforms raw simulation data into a practical, accessible, and visually rich platform that strengthens both decision-making and operational planning in the Registrar’s Office. By converting complex metrics into an intuitive real-world visualization, the system enables stakeholders to recognize inefficiencies and potential solutions more quickly. It also enhances communication among decision-makers by presenting technical findings in a format that non-technical users can easily interpret. Ultimately, the integration of DES and CS outputs into a 3D environment not only improves analytical accuracy but also fosters a more informed, proactive, and strategic approach to managing student flow and optimizing registrar services.

\subsubsection{4.3.2 3D Prototype Interface Guide}
 The \ref{fig:tenthFig} presented the 3D Prototype’s Interface and Visual Guide, which was designed to help users navigate and interpret the simulation environment effectively. It outlines the visual state of students, office layout elements, timed simulation modes, and navigation controlled integrated into the system.
 
\begin{figure}[H]
    \centering
    \includegraphics[width=10 cm\linewidth]{figures/student state and visual guide.png}
    \caption{Student State and Visual Guide}
    \label{fig:tenthFig}
\end{figure}

The Student State and Visual Guide section identifies how different elements in the simulation are represented. Students in the waiting line (DES) were shown in blue, those being served are highlighted in yellow, and those exiting after service appeared in green. CS students are indicated by red bases to distinguish them from DES entities. Additional visual indicators, such as red for the "waiting" status and gold for "active service," helped users easily monitor real-time changes in the system. The office layout used blue carpets for the waiting area, beige floors for service counters, and red queued management posts to clearly visualize flow paths.

The Navigation and Time Controls section explained how users interact with the simulation. Mouse control allowed for flexible viewing through rotation, zooming, and panning, while playback controls enabled starting, pausing, or scrubbing through the simulation timeline. A comparison feature allowed toggling between DES and CS students for side-by-side evaluation. Users could switch between Realistic Mode (real-time simulation) for live demonstrations and Fast Mode for analytical testing, making the prototype suitable for both trained operational staff and research analysis. This combination of visual clarity and interactive controls enhanced the usability and practical value of the 3D prototype.

\subsubsection{4.3.3 What-If Scenario Configuration and Testing}
To further enhance the functionality of the simulation platform, a What-If Scenario feature was integrated to allow administrators to test various operational strategies and their impact on system performance. This tool supported data-driven decision-making by simulating different configurations, such as changes in service capacity, operating hours, or arrival patterns, before applying them in real operations. \ref{fig:elevenFig} displayed the What-If Scenario Configuration Interface.

\begin{figure}[H]
    \centering
    \includegraphics[width=10 cm\linewidth]{figures/Screenshot 2025-10-05 101350.png}
    \caption{What-If Scenario Configuration Interface}
    \label{fig:elevenFig}
\end{figure}

The image above illustrated the What-If Scenario interface, which offered multiple configuration options: adding more counters, speeding up service times, simulating peak hour surges, and extending office hours. Each scenario could have been tested with a single click, triggering the simulation engine to recalculate key performance metrics such as average weighted time, throughput, utilization, and overall efficiency. For example, selecting the “Faster Service” scenario reduced service time by 30\%, allowed administrators to immediately observe the resulting decrease in wait time and increase in throughput in the Scenario Results panel. This interactive feature enables the Registrar’s Office to explore different operational strategies, anticipate outcomes, and choose the most effective improvements without disrupting real-world operations.

\begin{refsection}
%=======================================================%
%%%%% Do not delete this part %%%%%%
\clearpage

\printbibliography[heading=subbibintoc, title={\texorpdfstring{\centering}{} Notes}]
\end{refsection}

    \chapter{Summary, Conclusions, and Recommendations}
\begin{refsection}
The next chapter summarizes these findings objectively, presents conclusions, and makes targeted recommendations for the registrar’s office. The results from Chapter 4 are synthesized here to highlight the implications of the comparative analysis between the Discrete-Event Simulation (DES) and Continuous Simulation (CS) models, as well as the role of the 3D prototype in supported decision-making in the Registrar’s Office.

\section{Summary}
In this study, the emphasis is on improving student movement and service flow in the Registrar’s Office of Camarines Sur Polytechnic Colleges using both Discrete-Event Simulation (DES) and Continuous Simulation (CS). Moreover, the research aims to develop a 3D prototype that could improve stakeholders’ understanding of office operations. Through the comparison of both models, the study identifies strengths and limitations in managing student flow. These insights support informed decisions for improving efficiency and service delivery in the Registrar’s Office.

The Discrete Event Simulation was utilized to model the individual student process activities as separate events such as arrivals, queuing, and service completion. Using separate models of students, DES simulated real-world variability of student arrivals, service, and queueing in the Registrar’s Office effectively. This application of DES enabled the accurate analysis of bottlenecks and resource waste, ensuring that DES is very effective in the short term for maximizing day-to-day operations.

A Continuous Simulation model was developed to study the total student movement as a continuing process. CS dealt with student service demands, congestions, and resource allocations in terms of trends. Though it was effective for studying trends for long-term planning purposes, it was not effective enough for capturing discontinuous trends in queueing that occur in peak registration times. These were dealt with in the Queueing Model.

It is clearly evident that the difference in the performance of DES and CS is significant  based on key metrics such as average waiting time, queue behavior, service utilization, throughput, and overall system efficiency. The DES model recorded a higher average waiting time of 59.60 minutes, reflecting its sensitivity to peak-time congestion and individual delays, while the CS model produced a smoother and lower average waiting time of 12.00 minutes due to its flow-based averaging mechanism. These results demonstrate that DES provides a more realistic depiction of actual student experience during busy operational periods.
In terms of efficiency and output, DES achieved a higher throughput of 21.9 students per hour with a service utilization rate of 94.3\%, indicating effective use of service counters and accurate modeling of workload intensity. In contrast, the CS model recorded a throughput of 12.6 students per hour and a service utilization of 49.5\%, suggesting under utilization caused by the smoothing effect of continuous modeling. Overall system efficiency further favored DES at 34.5\%, compared to 19.5\%for CS, confirming DES as more suitable for operational decision-making, while CS remains valuable for understanding macro-level system behavior.

A 3D prototype system was created to enable the graphical illustration of the results of both the DES and CS models for the simulation system. This was achieved using the 3D setting, which converted the numerical results of the simulation into a graphical representation of the Registrar’s Office system, illustrating student flows, queues, and congestions. The 3D prototype explains that it allows for data-informed decision-making and more effective operational planning by facilitating tests of what-if questions, analysis of layout alternatives, and analysis of staff allocation proposals before implementation takes place.
Again, through the integration of DES, CS, and 3D, the implementation process for student services is streamlined, with more effective decision-making through evidence. 

\section{Findings}
These are the results submitted by the researchers with respect to the objectives of the study:


\begin{enumerate}
    \item The study successfully implemented both Discrete-Event Simulation (DES) and Continuous Simulation (CS) models using actual datasets from the Registrar’s Office, which included arrival times, service durations, queue lengths, and counter assignments. The DES model effectively captured individual student transactions, queues, and service variations, providing a detailed representation of system behavior. Meanwhile, the CS model generated a broader and smoother overview of student flow trends, emphasizing general system dynamics rather than micro-level fluctuations. Together, the two models offered complementary perspectives in understanding the operational flow of registrar services.
    \item The comparison was guided by key performance metrics that illustrate how each model handled student flow, detailed as follows:
    \begin{enumerate} 
    \item [a.)]Average Waiting Time

    The DES model recorded an average waiting time of 59.60 minutes, significantly higher than the CS model’s 12.00 minutes. This difference shows how DES captures individual process delays and peak-time congestion more realistically, while CS produces an averaged behavior that smooths out fluctuations in student arrivals.

   \item [b.)]Queue Length

     Findings indicated that DES represented longer and more variable queue formations compared to CS. The CS model displayed shorter queues due to its continuous nature, which distributes system load more uniformly. This difference highlights DES’s ability to reflect real-time bottlenecks more accurately, especially during high-traffic periods.

    \item [c.)] Service Utilization

   DES demonstrated a higher resource utilization of 94.3\%, showing that service counters were active for most of the simulated time. Meanwhile, CS recorded only 49.5\%, indicating underutilization due to the smoothing effect of continuous flow modeling. This suggests that DES is more effective at capturing actual workload intensity experienced by front-line personnel.

    \item [d.)]Throughput

    DES achieved a throughput of 21.9 students per hour, outperforming CS, which processed 12.6 students per hour. This implies that DES more accurately reflects real-world operational output because it tracks distinct service completions, whereas CS models output as a flow rate rather than discrete transactions.

   \item [e.)]System Efficiency

    Overall system efficiency further highlighted DES’s advantage, achieving 34.5\%, compared to 19.5\% for CS. DES showed better performance in capturing idle time, congestion points, and system responsiveness. Although CS offered long-term patterns, it was less capable of representing actual operational pressures.    
\end{enumerate}

    \item A 3D prototype system was developed to visualize the results of the simulation models. This prototype provided a realistic representation of the Registrar’s Office and showed student movements, queue formations, and counter utilization. It allowed administrators to observe congestion points, test scenarios such as adding service counters, and evaluate strategies for reducing waiting time. By transforming quantitative results into a visual and interactive format, the prototype made complex data more accessible and actionable, enhancing decision-making and operational planning within the Registrar’s Office.
\end{enumerate}

\section{Conclusion}
Hence, based on the findings, the researchers concluded:

\begin{enumerate}
 \item The study concluded that both Discrete-Event Simulation (DES) and Continuous Simulation (CS) effectively modeled student flow in the Registrar’s Office, but they provided different levels of detail. DES offered a precise, event-driven representation of arrivals, queues, and service processes, making it highly suitable for operational analysis. In contrast, CS provided a smoother and more generalized view of system behavior, which was better suited for identifying long-term flow patterns. Together, these models demonstrated the value of simulation as a tool for understanding and optimizing registrar operations.

 \item The evaluation confirmed that DES outperformed CS in terms of efficiency, accuracy, and responsiveness. DES consistently produced shorter waiting times, smaller queue lengths, higher throughput, and better resource utilization, demonstrating its ability to capture real-world variations and minimize congestion. CS, while still effective, showed longer waiting times and slightly less efficient queue management, reflecting its limitation in capturing abrupt fluctuations. Therefore, DES was considered more appropriate for day-to-day operational improvements, while CS provided valuable insights for strategic and long-term planning.

 \item The study further concluded that the development of the 3D prototype system enhanced the applicability of the simulation results by transforming numerical data into a visual and interactive tool. This prototype allowed administrators to easily observe student movements, counter utilization, and congestion points, making simulation outputs more accessible to decision-makers. It also provided a practical platform for testing different scenarios and strategies, thereby supporting data-driven decisions in managing registrar operations.
\end{enumerate}

\section{Recommendations}
On the basis of the findings, the researchers recommend:

\begin{enumerate}
    \item Discrete-Event Simulation (DES) and Continuous Simulation (CS) should be continuously applied in analyzing registrar operations. DES is recommended for addressing immediate service concerns such as congestion, waiting time, and counter utilization due to its detailed and event-driven nature, making it suitable for daily operational management. Meanwhile, CS should be used for long-term forecasting and strategic planning, including enrollment trends and future resource requirements. The development of hybrid simulation models that integrate DES with forecasting techniques may further enhance accuracy and system responsiveness.

    \item The outputs of DES have to be the main reference in trying to minimize queue length, waiting time, and enhance the daily service efficiency since it gave detailed information about the transactions of each student and real-time system behavior. On the other hand, outputs from CS should drive long-term staffing decisions, counter allocation, and resource planning for sustainability of operational efficiency. Monitoring and evaluation of key performance indicators, such as average waiting time, queue length, throughput, and utilization—should be institutionalized to support continuous improvement.

    \item The developed 3D prototype system should be further enhanced, regularly maintained, and gradually expanded to other CSPC service units with similar congestion issues, such as the Cashier and Library. The system should be used to simulate various “what-if” scenarios, including changes in staffing levels, counter layouts, and student arrival patterns, to assess potential effects before implementation. 
    Additionally, the 3D prototype serve as a improvement of service workflows, supporting informed decision-making and effective operational planning.
    
\end{enumerate}
%=======================================================%
%%%%% Do not delete this part %%%%%%
\clearpage

\printbibliography[heading=subbibintoc, title={\texorpdfstring{\centering}{} Notes}]
\end{refsection}
    \makeBibliography
    
% The environment used here (theappendices) is a wrapper for the basic appendices environment which changes the appearance of the title page and the structure and appearance of the appendices in the table of contents and PDF bookmarks. The original functionality can be restored by simply removing the 'the' from the \begin{} and \end{} statements below.

\begin{theappendices}   

\appendix

\chapter{Source Codes}
\centering
\textbf{Constraint}
\\
\begin{verbatim}
from constraint import *

from playground.models import Instructor
from playground.models import Laboratory
from playground.models import Schedule

from datetime import datetime


class SchedulingProblem:
    def __init__(
        self,
        course,
        section,
        instructor=None,
        laboratory=None,
        time=None,
        day=None,
    ):
        self.problem = Problem(BacktrackingSolver())
        self.course = course
        self.section = section

        # These constraints are hard constraints if they are defined
        self.laboratory = laboratory
        self.time = time
        self.day = day
        self.instructor = instructor

        self.timeslots = [
            "Monday 7:00 am",
            "Monday 8:00 am",
            "Monday 9:00 am",
            "Monday 10:00 am",
            "Monday 11:00 am",
            "Monday 12:00 pm",
            "Monday 1:00 pm",
            "Monday 2:00 pm",
            "Monday 3:00 pm",
            "Monday 4:00 pm",
            "Monday 5:00 pm",
            "Monday 6:00 pm",
            "Monday 7:00 pm",
            "Tuesday 7:00 am",
            "Tuesday 8:00 am",
            "Tuesday 9:00 am",
            "Tuesday 10:00 am",
            "Tuesday 11:00 am",
            "Tuesday 12:00 pm",
            "Tuesday 1:00 pm",
            "Tuesday 2:00 pm",
            "Tuesday 3:00 pm",
            "Tuesday 4:00 pm",
            "Tuesday 5:00 pm",
            "Tuesday 6:00 pm",
            "Tuesday 7:00 pm",
            "Wednesday 7:00 am",
            "Wednesday 8:00 am",
            "Wednesday 9:00 am",
            "Wednesday 10:00 am",
            "Wednesday 11:00 am",
            "Wednesday 12:00 pm",
            "Wednesday 1:00 pm",
            "Wednesday 2:00 pm",
            "Wednesday 3:00 pm",
            "Wednesday 4:00 pm",
            "Wednesday 5:00 pm",
            "Wednesday 6:00 pm",
            "Wednesday 7:00 pm",
            "Thursday 7:00 am",
            "Thursday 8:00 am",
            "Thursday 9:00 am",
            "Thursday 10:00 am",
            "Thursday 11:00 am",
            "Thursday 12:00 pm",
            "Thursday 1:00 pm",
            "Thursday 2:00 pm",
            "Thursday 3:00 pm",
            "Thursday 4:00 pm",
            "Thursday 5:00 pm",
            "Thursday 6:00 pm",
            "Thursday 7:00 pm",
            "Friday 7:00 am",
            "Friday 8:00 am",
            "Friday 9:00 am",
            "Friday 10:00 am",
            "Friday 11:00 am",
            "Friday 12:00 pm",
            "Friday 1:00 pm",
            "Friday 2:00 pm",
            "Friday 3:00 pm",
            "Friday 4:00 pm",
            "Friday 5:00 pm",
            "Friday 6:00 pm",
            "Friday 7:00 pm",
            "Saturday 7:00 am",
            "Saturday 8:00 am",
            "Saturday 9:00 am",
            "Saturday 10:00 am",
            "Saturday 11:00 am",
            "Saturday 12:00 pm",
            "Saturday 1:00 pm",
            "Saturday 2:00 pm",
            "Saturday 3:00 pm",
            "Saturday 4:00 pm",
            "Saturday 5:00 pm",
            "Saturday 6:00 pm",
            "Saturday 7:00 pm",
        ]

    def solve(self):
        labs = Laboratory.objects.all()
        instructors = Instructor.objects.all()

        self.problem.addVariable("laboratory", labs)
        self.problem.addVariable("timeslot", self.timeslots)
        self.problem.addVariable("instructors", instructors)

        self.problem.addConstraint(self.laboratory_constraint,
        ["laboratory", "timeslot"])
        self.problem.addConstraint(self.time_constraint, 
        
        ["timeslot"])
        self.problem.addConstraint(self.instructor_constraint, 
        ["instructors", "timeslot"])

        solutions = self.problem.getSolutions()
        if len(solutions) == 0:
            self.problem.reset()
            self.problem.addVariable("timeslot", self.timeslots)
            self.problem.addVariable("laboratory", labs)
            self.problem.addVariable("instructors", instructors)
            self.problem.addConstraint(self.instructor_constraint,
            ["instructors", "timeslot"])
            self.problem.addConstraint
            (self.laboratory_soft_constraint, 
            
            ["laboratory", "timeslot"])
            self.problem.addConstraint
            (self.time_constraint, ["timeslot"])
            solutions = self.problem.getSolutions()

        return solutions
    

    def instructor_constraint(self, instructor,timeslot):
        if (
            instructor.expertise is not None
            and instructor.expertise.name == self.course
        ):
            instructor_ids = Instructor.objects.filter
            (name=instructor.name)
            time = self.get_time(timeslot)
            
            for instructor in instructor_ids:
                schedules = instructor.get_schedules()
                if not schedules:
                    return True
                
                for schedule in schedules:
                    start = self.get_time(schedule.time_start)
                    end = self.get_time(schedule.time_end)
                    
                    if (
                        time >= start
                        and time < end
                        and timeslot.split(" ")[0] == schedule.
                        time_start.split(" ")[0]
                    ):
                        return False
                    else:
                        return True

    def laboratory_soft_constraint(self, laboratory, timeslot):
        if self.laboratory is not None:
            return self.laboratory == laboratory.name
        special_courses = (
            laboratory.course.replace("[", "")
            .replace("]", "")
            .replace("'", "")
            .split(", ")
        )
        if special_courses[0] == "":
            return True
        
        time = self.get_time(timeslot)
            
        schedules = laboratory.get_schedules()
        if not schedules:
            return True
        
        for schedule in schedules:
            start = self.get_time(schedule.time_start)
            end = self.get_time(schedule.time_end)
            
            if (
                time >= start
                and time < end
                and timeslot.split(" ")[0] == 
                schedule.time_start.split(" ")[0]
            ):
                return False
            else:
                return True
        

    def laboratory_constraint(self, laboratory, timeslot):
        if self.laboratory is not None:
            return self.laboratory == laboratory.name
        special_courses = (
            laboratory.course.replace("[", "")
            .replace("]", "")
            .replace("'", "")
            .split(", ")
        )
        schedules = laboratory.get_schedules()
        time = self.get_time(timeslot)
        
        if self.course not in special_courses:
            return False
        for schedule in schedules:
            start = self.get_time(schedule.time_start)
            end = self.get_time(schedule.time_end)
            
            if (
                time >= start
                and time < end
                and timeslot.split(" ")[0] == 
                schedule.time_start.split(" ")[0]
            ):
                return False
            else:
                return True
        
        return True

    def time_constraint(self, timeslot):
        schedules = Schedule.objects.filter(section=self.section).
        all()
        time = self.get_time(timeslot)

        for schedule in schedules:
            start = self.get_time(schedule.time_start)
            end = self.get_time(schedule.time_end)
            if (
                time >= start
                and time < end
                and timeslot.split(" ")[0] == schedule.time_start.
                split(" ")[0]
            ):
                return False

        return True

    def get_time(self, time):
        time_format = "%I:%M %p"
        time_arr = time.split(" ")
        time_arr.pop(0)
        time = " ".join(time_arr)
        value = datetime.strptime(time, time_format).time()

        return value

\end{verbatim}

\begin{table}[!h]
\centering
\chapter{DATASET}
\label{tbl:sampleTbl4}
\resizebox{\textwidth}{!}{%
\begin{tabular}{c|c|c|c|c|c|c|c|c}
\hline
\textbf{Date} & \textbf{Student ID} & \textbf{Arrival Time} &
\textbf{Service Start} & \textbf{Service End} &
\textbf{Service Duration(min)} & \textbf{Purpose} & \textbf{Queue Length} & \textbf{Counter ID (RECEIVING)}\\ \hline
10/13/25 & S1  & 6:07 am  & 8:00 am  & 8:03 am  & 3 & TOR            & 15 & C1 \\
10/13/25 & S2  & 6:10 am  & 8:03 am  & 8:06 am  & 3 & Diploma        & 14 & C1 \\
10/13/25 & S3  & 6:25 am  & 8:06 am  & 8:09 am  & 3 & CAV            & 13 & C1 \\
10/13/25 & S4  & 6:32 am  & 8:09 am  & 8:12 am  & 3 & Authentication & 12 & C1 \\
10/13/25 & S5  & 6:39 am  & 8:12 am  & 8:15 am  & 3 & TOR            & 11 & C1 \\
10/13/25 & S6  & 6:45 am  & 8:15 am  & 8:18 am  & 3 & Diploma        & 10 & C1 \\
10/13/25 & S7  & 6:50 am  & 8:18 am  & 8:20 am  & 2 & TOR            & 9  & C1 \\
10/13/25 & S8  & 6:54 am  & 8:20 am  & 8:22 am  & 2 & CAV            & 9  & C1 \\
10/13/25 & S9  & 6:57 am  & 8:22 am  & 8:23 am  & 1 & Authentication & 8  & C1 \\
10/13/25 & S10 & 7:05 am  & 8:23 am  & 8:26 am  & 3 & TOR            & 7  & C1 \\
10/13/25 & S11 & 7:15 am  & 8:26 am  & 8:29 am  & 3 & Diploma        & 6  & C1 \\
10/13/25 & S12 & 7:35 am  &  8:29am  & 8:32 am  & 3 & CAV            & 6  & C1 \\
10/14/25 & S1  & 7:23 am  & 8:00 am  & 8:03 am  & 3 & Diploma        & 3  & C1 \\
10/14/25 & S2  & 7:36 am  & 8:03 am  & 8:05 am  & 2 & TOR            & 2  & C1 \\
10/14/25 & S3  & 7:56 am  & 8:05 am  & 8:07 am  & 2 & Authentication & 1  & C1 \\
10/14/25 & S4  & 8:16 am  & 9:31 am  & 9:34 am  & 3 & Authentication & 7  & C1 \\
10/14/25 & S5  & 8:23 am  & 9:34 am  & 9:35 am  & 1 & Authentication & 6  & C1 \\
10/14/25 & S6  & 8:32 am  & 9:35 am  & 9:37 am  & 2 & TOR            & 5  & C1 \\
10/14/25 & S7  & 8:37 am  & 9:37 am  & 9:39 am  & 2 & TOR            & 4  & C1 \\
10/14/25 & S8  & 9:06 am  & 9:39 am  & 9:42 am  & 3 & TOR            & 3  & C1 \\
10/14/25 & S9  & 9:16 am  & 9:42 am  & 9:44 am  & 2 & Diploma        & 2  & C1 \\
10/14/25 & S10 & 9:20 am  & 9:44 am  & 9:47 am  & 3 & TOR            & 1  & C1 \\
10/14/25 & S11 & 10:12 am & 10:13 am & 10:15 am & 2 & Diploma        & 1  & C1 \\
10/14/25 & S12 & 10:29 am & 10:30 am & 10:32 am & 2 & CAV            & 1  & C1 \\
10/14/25 & S13 & 11:00 am & 11:01 am & 11:03 am & 2 & TOR            & 1  & C1 \\
10/14/25 & S14 & 11:28 am & 11:29 am & 11:31 am & 2 & CAV            & 1  & C1 \\
10/14/25 & S15 & 11:31 am & 11:31 am & 11:34 am & 3 & TOR            & 1  & C1 \\
10/14/25 & S16 & 11:36 am & 11:37 am & 11:40 am & 3 & TOR            & 1  & C1 \\
10/14/25 & S17 & 11:45 am & 11:46 am & 11:49 am & 3 & Diploma        & 1  & C1 \\
10/15/25 & S1  & 7:22 am  & 8:00 am  & 8:02 am  & 2 & Authentication & 5  & C1 \\
10/15/25 & S2  & 7:28 am  & 8:02 am  & 8:04 am  & 2 & CAV            & 4  & C1 \\
10/15/25 & S3  & 7:34 am  & 8:04 am  & 8:06 am  & 2 & Diploma        & 3  & C1 \\
10/15/25 & S4  & 7:47 am  & 8:06 am  & 8:09 am  & 3 & Diploma        & 2  & C1 \\
10/15/25 & S5  & 7:54 am  & 8:09 am  & 8:12 am  & 3 & TOR            & 1  & C1 \\
10/15/25 & S6  & 8:10 am  & 8:12 am  & 8:15 am  & 3 & CAV            & 1  & C1 \\
10/15/25 & S7  & 8:25 am  & 8:29 am  & 8:31 am  & 3 & Diploma        & 1  & C1 \\
10/15/25 & S8  & 8:34 am  & 8:42 am  & 8:44 am  & 2 & CAV            & 1  & C1 \\
10/15/25 & S9  & 8:59 am  & 9:04 am  & 9:06 am  & 2 & Diploma        & 2  & C1 \\
10/15/25 & S10 & 9:00 am  & 9:09 am  & 9:12 am  & 2 & CAV            & 1  & C1 \\
10/15/25 & S11 & 9:12 am  & 9:16 am  & 9:18 am  & 2 & TOR            & 1  & C1 \\
10/15/25 & S12 & 9:18 am  & 9:26 am  & 9:28 am  & 2 & TOR            & 1  & C1 \\
10/15/25 & S13 & 10:11 am & 10:16 am & 10:18 am & 2 & Diploma        & 1  & C1 \\
10/15/25 & S14 & 10:18 am & 10:21 am & 10:23 am & 2 & CAV            & 1  & C1 \\
10/16/25 & S1  & 7:08 am  & 8:02 am  & 8:04 am  & 2 & Diploma        & 2  & C1 \\
10/16/25 & S2  & 7:20 am  & 8:06 am  & 8:08 am  & 2 & TOR            & 1  & C1 \\
10/16/25 & S3  & 8:35 am  & 8:46 am  & 8:48 am  & 2 & TOR            & 1  & C1 \\
10/16/25 & S4  & 8:53 am  & 8:58 am  & 8:59 am  & 1 & TOR            & 1  & C1 \\
10/16/25 & S5  & 9:03 am  & 9:19 am  & 9:21 am  & 2 & CAV            & 1  & C1 \\
10/16/25 & S6  & 9:29 am  & 9:34 am  & 9:36 am  & 2 & CAV            & 1  & C1 \\
10/16/25 & S7  & 9:55 am  & 10:01 am & 10:03 am & 2 & CAV            & 1  & C1 \\
10/16/25 & S8  & 10:17 am & 10:24 am & 10:27 am & 3 & TOR            & 2  & C1 \\
10/16/25 & S9  & 10:23 am & 10:35 am & 10:36 am & 1 & CAV            & 1  & C1 \\
10/16/25 & S10 & 11:10 am & 11:18 am & 11:21 am & 3 & TOR            & 1  & C1 \\
10/16/25 & S11 & 11:19 am & 11:26 am & 11:28 am & 2 & CAV            & 1  & C1 \\
10/16/25 & S12 & 11:41 am & 11:47 am & 11:48 am & 1 & CAV            & 1  & C1 \\
10/16/25 & S13 & 11:51 am & 11:58 am & 11:59 am & 1 & CAV            & 1  & C1 \\
10/16/25 & S14 & 12:18 pm & 12:25 pm & 12:27 am & 2 & CAV            & 2  & C1 \\
10/16/25 & S15 & 12:25 am & 12:33 pm & 12:35 pm & 2 & CAV            & 1  & C1 \\
10/16/25 & S16 & 12:49 pm & 12:52 pm & 12:54 pm & 2 & Authentication & 1  & C1 \\
10/17/25 & S1  & 7:14 am  & 8:05 am  & 8:07 am  & 2 & CAV            & 5  & C1 \\
10/17/25 & S2  & 7:22 am  & 8:12 am  & 8:14 am  & 2 & Diploma        & 4  & C1 \\
10/17/25 & S3  & 7:34 am  & 8:15 am  & 8:17 am  & 2 & Diploma        & 3  & C1 \\
10/17/25 & S4  & 7:41 am  & 8:23 am  & 8:25 am  & 2 & TOR            & 2  & C1 \\
10/17/25 & S5  & 7:52 am  & 8:32 am  & 8:33 am  & 1 & CAV            & 1  & C1 \\
10/17/25 & S6  & 8:07 am  & 8:45 am  & 8:47 am  & 2 & CAV            & 0  & C1 \\
10/17/25 & S7  & 8:19 am  & 8:51 am  & 8:52 am  & 1 & CAV            & 0  & C1 \\ \hline
\end{tabular}%
}
\end{table}


\begin{table}[ht]
\centering
\label{tbl:sampleTbl4}
\resizebox{\textwidth}{!}{%
\begin{tabular}{c|c|c|c|c|c|c|c|c}
\hline
\textbf{Date} & \textbf{Student ID} & \textbf{Arrival Time} &
\textbf{Service Start} & \textbf{Service End} &
\textbf{Service Duration(min)} & \textbf{Purpose} & \textbf{Queue Length} & \textbf{Counter ID (RELEASING)}\\ \hline
10/13/25 & S1  & 8:40 am  & 10:30 am  & 10:32 am & 2 & Authentication & 30 & C2\\
10/13/25 & S2  & 8:46 am  & 10:32 am  & 10:35 am & 3 & CAV            & 30 & C2\\
10/13/25 & S3  & 8:51 am  & 10:35 am  & 10:38 am & 3 & Diploma        & 30 & C2\\
10/13/25 & S4  & 8:56 am  & 10:38 am  & 10:41 am & 3 & CAV            & 31 & C2\\
10/13/25 & S5  & 9:01 am  & 10:41 am  & 10:43 am & 2 & TOR            & 31 & C2\\
10/13/25 & S6  & 9:10 am  & 10:43 am  & 10:45 am & 2 & CAV            & 30 & C2\\
10/13/25 & S7  & 9:14 am  & 10:45 am  & 10:47 am & 2 & CAV            & 29 & C2\\
10/13/25 & S8  & 9:20 am  & 10:47 am  & 10:49 am & 2 & CAV            & 28 & C2\\
10/13/25 & S9  & 9:23 am  & 10:49 am  & 10:51 am & 2 & Diploma        & 27 & C2\\
10/13/25 & S10 & 9:27 am  & 10:51 am  & 10:53 am & 2 & TOR            & 26 & C2\\
10/13/25 & S11 & 9:36 am  & 10:53 am  & 10:55 am & 2 & CAV            & 25 & C2\\
10/13/25 & S12 & 9:39 am  & 10:55 am  & 10:58 am & 3 & TOR            & 24 & C2\\
10/13/25 & S13 & 9:42 am  & 10:58 am  & 11:00 am & 2 & CAV            & 23 & C2\\
10/13/25 & S14 & 9:45 am  & 12:00 pm  & 12:03 pm & 3 & Authentication & 40 & C2\\
10/13/25 & S15 & 9:48 am  & 12:03 pm  & 12:05 pm & 2 & Authentication & 40 & C2\\
10/13/25 & S16 & 9:51 am  & 12:05 pm  & 12:07 pm & 2 & CAV            & 39 & C2\\
10/13/25 & S17 & 9:53 am  & 12:07 pm  & 12:10 pm & 3 & CAV            & 38 & C2\\
10/13/25 & S18 & 9:56 am  & 12:10 pm  & 12:13 pm & 3 & CAV            & 37 & C2\\
10/13/25 & S19 & 9:58 am  & 12:13 pm  & 12:16 pm & 3 & Diploma        & 36 & C2\\
10/13/25 & S20 & 9:59 am  & 12:16 pm  & 12:19 pm & 3 & Authentication & 35 & C2\\
10/13/25 & S21 & 10:01 am & 12:19 pm  & 12:21 pm & 2 & CAV            & 34 & C2\\
10/13/25 & S22 & 10:03 am & 12:21 pm  & 12:23 pm & 2 & CAV            & 33 & C2\\
10/13/25 & S23 & 10:06 am & 12:23 pm  & 12:25 pm & 2 & Diploma        & 32 & C2\\
10/13/25 & S24 & 10:09 am & 12:25 pm  & 12:27 pm & 2 & Authentication & 31 & C2\\
10/13/25 & S25 & 10:11 am & 12:27 pm  & 12:30 pm & 3 & Diploma        & 30 & C2\\
10/13/25 & S26 & 10:16 am & 1:30 pm   & 1:32 pm  & 2 & Diploma        & 31 & C2\\
10/13/25 & S27 & 10:18 am & 1:32 pm   & 1:34 pm  & 2 & TOR            & 30 & C2\\
10/13/25 & S28 & 10:22 am & 1:34 pm   & 1:36 pm  & 2 & TOR            & 29 & C2\\
10/13/25 & S29 & 10:25 am & 1:36 pm   & 1:38 pm  & 2 & CAV            & 28 & C2\\
10/13/25 & S30 & 10:28 am & 1:38 pm   & 1:40 pm  & 2 & Authentication & 27 & C2\\
10/13/25 & S31 & 10:30 am & 1:40 pm   & 1:43 pm  & 3 & CAV            & 26 & C2\\
10/13/25 & S32 & 10:33 am & 1:43 pm   & 1:46 pm  & 3 & CAV            & 25 & C2\\
10/13/25 & S33 & 10:36 am & 1:46 pm   & 1:49 pm  & 3 & CAV            & 24 & C2\\
10/13/25 & S34 & 10:37 am & 1:49 pm   & 1:52 pm  & 3 & Diploma        & 23 & C2\\
10/13/25 & S35 & 10:39 am & 1:52 pm   & 1:55 pm  & 3 & CAV            & 22 & C2\\
10/13/25 & S36 & 11:00 am & 1:55 pm   & 1:58 pm  & 3 & Authentication & 21 & C2\\
10/13/25 & S37 & 11:02 am & 1:58 pm   & 2:00 pm  & 2 & Authentication & 20 & C2\\
10/13/25 & S38 & 11:06 am & 3:00 pm   & 3:03 pm  & 3 & Authentication & 19 & C2\\
10/13/25 & S39 & 11:08 am & 3:03 pm   & 3:06 pm  & 3 & Diploma        & 18 & C2\\
10/13/25 & S40 & 11:11 am & 3:06 pm   & 3:08 pm  & 2 & CAV            & 17 & C2\\
10/13/25 & S41 & 11:13 am & 3:08 pm   & 3:10 pm  & 2 & Authentication & 16 & C2\\
10/13/25 & S42 & 11:17 am & 3:10 pm   & 3:12 pm  & 2 & TOR            & 15 & C2\\
10/13/25 & S43 & 11:19 am & 3:12 pm   & 3:14 pm  & 2 & CAV            & 14 & C2\\
10/13/25 & S44 & 11:23 am & 3:14 pm   & 3:16 pm  & 2 & TOR            & 13 & C2\\
10/13/25 & S45 & 11:29 am & 3:16 pm   & 3:18 pm  & 2 & CAV            & 12 & C2\\
10/13/25 & S46 & 11:38 am & 3:18 pm   & 3:20 pm  & 2 & TOR            & 11 & C2\\
10/14/25 & S1  & 6:08 am  & 10:30 am  & 10:32 am & 2 & CAV            & 36 & C2\\
10/14/25 & S2  & 6:12 am  & 10:32 am  & 10:34 am & 2 & CAV            & 35 & C2\\
10/14/25 & S3  & 6:17 am  & 10:34 am  & 10:36 am & 2 & CAV            & 34 & C2\\
10/14/25 & S4  & 6:24 am  & 10:36 am  & 10:39 am & 3 & CAV            & 34 & C2\\
10/14/25 & S5  & 6:27 am  & 10:39 am  & 10:42 am & 3 & CAV            & 33 & C2\\
10/14/25 & S6  & 6:34 am  & 10:42 am  & 10:44 am & 2 & Diploma        & 32 & C2\\
10/14/25 & S7  & 6:38 am  & 10:44 am  & 10:46 am & 2 & CAV            & 31 & C2\\
10/14/25 & S8  & 6:43 am  & 10:46 am  & 10:48 am & 2 & CAV            & 31 & C2\\
10/14/25 & S9  & 6:54 am  & 10:48 am  & 10:50 am & 2 & CAV            & 30 & C2\\
10/14/25 & S10 & 7:01 am  & 10:50 am  & 10:52 am & 2 & CAV            & 29 & C2\\
10/14/25 & S11 & 7:06am   & 10:52 am  & 10:55 am & 3 & CAV            & 28 & C2\\
10/14/25 & S12 & 7:10 am  & 10:55 am  & 10:58 am & 3 & CAV            & 27 & C2\\
10/14/25 & S13 & 7:12 am  & 10:58 am  & 11:00 am & 2 & CAV            & 26 & C2\\
10/14/25 & S14 & 7:27 am  & 12:00 pm  & 12:03 pm & 3 & TOR            & 25 & C2\\
10/14/25 & S15 & 7:32 am  & 12:03 pm  & 12:06 pm & 3 & TOR            & 24 & C2\\
10/14/25 & S16 & 7:36 am  & 12:06 pm  & 12:09 pm & 3 & CAV            & 23 & C2\\
10/14/25 & S17 & 7:41 am  & 12:09 pm  & 12:11 pm & 2 & CAV            & 22 & C2\\
10/14/25 & S18 & 7:46 am  & 12:11 pm  & 12:13 pm & 2 & CAV            & 21 & C2\\
10/14/25 & S19 & 7:51 am  & 12:13 pm  & 12:15 pm & 2 & CAV            & 20 & C2\\
10/14/25 & S20 & 7:56 am  & 12:15 pm  & 12:17 pm & 2 & TOR            & 19 & C2\\
10/14/25 & S21 & 8:05 am  & 12:17 pm  & 12:19 pm & 2 & CAV            & 18 & C2\\
10/14/25 & S22 & 8:08 am  & 12:19 pm  & 12:21 pm & 2 & TOR            & 17 & C2\\
10/14/25 & S23 & 8:17 am  & 12:21 pm  & 12:24 pm & 3 & Diploma        & 16 & C2\\
10/14/25 & S24 & 8:30 am  & 12:24 pm  & 12:27 pm & 3 & Diploma        & 15 & C2\\
10/14/25 & S25 & 8:42 am  & 12:27 pm  & 12:30 pm & 3 & TOR            & 14 & C2\\
10/14/25 & S26 & 8:48 am  & 1:30 pm   & 1:33 pm  & 3 & Diploma        & 13 & C2\\
10/14/25 & S27 & 8:59 am  & 1:33 pm   & 1:36 pm  & 3 & CAV            & 12 & C2\\
10/14/25 & S28 & 9:00 am  & 1:36 pm   & 1:39 pm  & 3 & CAV            & 11 & C2\\ \hline
\end{tabular}%
}
\end{table}


\begin{table}[ht]
\centering
\resizebox{\textwidth}{!}{%
\begin{tabular}{c|c|c|c|c|c|c|c|c}
\hline
\textbf{Date} & \textbf{Student ID} & \textbf{Arrival Time} &
\textbf{Service Start} & \textbf{Service End} &
\textbf{Service Duration(min)} & \textbf{Purpose} & \textbf{Queue Length} & \textbf{Counter ID (RELEASING)}\\ \hline
10/15/25 & S1  & 6:21 am  & 10:30 am  & 10:32 am & 2 & CAV            & 36 & C2\\
10/15/25 & S2  & 6:28 am  & 10:32 am  & 10:34 am & 2 & CAV            & 35 & C2\\
10/15/25 & S3  & 6:34 am  & 10:34 am  & 10:36 am & 2 & Diploma        & 34 & C2\\
10/15/25 & S4  & 6:37 am  & 10:36 am  & 10:39 am & 3 & CAV            & 33 & C2\\
10/15/25 & S5  & 6:47 am  & 10:39 am  & 10:42 am & 3 & TOR            & 32 & C2\\
10/15/25 & S6  & 6:52 am  & 10:42 am  & 10:44 am & 2 & CAV            & 31 & C2\\
10/15/25 & S7  & 6:57 am  & 10:44 am  & 10:47 am & 3 & CAV            & 30 & C2\\
10/15/25 & S8  & 7:00 am  & 10:47 am  & 10:50 am & 3 & CAV            & 29 & C2\\
10/15/25 & S9  & 7:04 am  & 10:50 am  & 10:53 am & 3 & Diploma        & 28 & C2\\
10/15/25 & S10 & 7:07 am  & 10:53 am  & 10:56 am & 3 & Authentication & 27 & C2\\
10/15/25 & S11 & 7:10 am  & 10:56 am  & 10:58 am & 2 & CAV            & 26 & C2\\
10/15/25 & S12 & 7:16 am  & 10:58 am  & 11:00 am & 2 & CAV            & 25 & C2\\
10/15/25 & S13 & 7:20 am  & 12:00 pm  & 12:02 pm & 2 & CAV            & 24 & C2\\
10/15/25 & S14 & 7:22 am  & 12:02 pm  & 12:04 pm & 2 & Authentication & 23 & C2\\
10/15/25 & S15 & 7:25 am  & 12:04 pm  & 12:06 pm & 2 & TOR            & 22 & C2\\
10/15/25 & S16 & 7:32 am  & 12:06 pm  & 12:08 pm & 2 & CAV            & 21 & C2\\
10/15/25 & S17 & 7:42 am  & 12:08 pm  & 12:10 pm & 2 & CAV            & 20 & C2\\
10/15/25 & S18 & 7:58 am  & 12:10 pm  & 12:12 pm & 2 & CAV            & 19 & C2\\
10/15/25 & S19 & 8:03 am  & 12:12 pm  & 12:15 pm & 3 & CAV            & 18 & C2\\
10/15/25 & S20 & 8:18 am  & 12:15 pm  & 12:18 pm & 3 & CAV            & 17 & C2\\
10/15/25 & S21 & 8:25 am  & 12:18 pm  & 12:21 pm & 3 & CAV            & 16 & C2\\
10/15/25 & S22 & 8:32 am  & 12:21 pm  & 12:24 pm & 3 & Diploma        & 15 & C2\\
10/15/25 & S23 & 8:34 am  & 12:24 pm  & 12:27 pm & 3 & Authentication & 14 & C2\\
10/15/25 & S24 & 9:00 am  & 12:27 pm  & 12:30 pm & 3 & TOR            & 13 & C2\\
10/15/25 & S25 & 9:05 am  & 1:30 pm   & 1:32 pm  & 2 & CAV            & 12 & C2\\
10/15/25 & S26 & 9:12 am  & 1:32 pm   & 1:34 pm  & 2 & Authentication & 11 & C2\\
10/15/25 & S27 & 9:23 am  & 1:34 pm   & 1:36 pm  & 2 & CAV            & 10 & C2\\
10/15/25 & S28 & 9:27 am  & 1:36 pm   & 1:38 pm  & 2 & CAV            & 9  & C2\\
10/16/25 & S1  & 7:15 am  & 10:30 am  & 10:32 am & 2 & CAV            & 21 & C2\\
10/16/25 & S2  & 7:18 am  & 10:32 am  & 10:34 am & 2 & CAV            & 20 & C2\\
10/16/25 & S3  & 7:21 am  & 10:34 am  & 10:36 am & 2 & Diploma        & 19 & C2\\
10/16/25 & S4  & 7:44 am  & 10:36 am  & 10:38 am & 2 & CAV            & 18 & C2\\
10/16/25 & S5  & 7:46 am  & 10:38 am  & 10:41 am & 3 & Diploma        & 17 & C2\\
10/16/25 & S6  & 7:51 am  & 10:41 am  & 10:44 am & 3 & CAV            & 16 & C2\\
10/16/25 & S7  & 8:02 am  & 10:44 am  & 10:46 am & 2 & TOR            & 15 & C2\\
10/16/25 & S8  & 8:17 am  & 10:46 am  & 10:48 am & 2 & TOR            & 14 & C2\\
10/16/25 & S9  & 8:23 am  & 10:48 am  & 10:50 am & 2 & CAV            & 13 & C2\\
10/16/25 & S10 & 8:45 am  & 10:50 am  & 10:52 am & 2 & TOR            & 12 & C2\\
10/16/25 & S11 & 8:47 am  & 10:52 am  & 10:54 am & 2 & CAV            & 11 & C2\\
10/16/25 & S12 & 8:56 am  & 10:54 am  & 10:56 am & 2 & CAV            & 10 & C2\\
10/16/25 & S13 & 8:59 am  & 10:56 am  & 10:58 am & 2 & CAV            & 9  & C2\\
10/16/25 & S14 & 9:00 am  & 10:58 am  & 11:00 am & 2 & CAV            & 8  & C2\\
10/16/25 & S15 & 9:19 am  & 12:00 pm  & 12:03 pm & 3 & TOR            & 7  & C2\\
10/16/25 & S16 & 9:20 am  & 12:03 pm  & 12:06 pm & 3 & TOR            & 6  & C2\\
10/17/25 & S1  & 6:12 am  & 10:30 am  & 10:33 am & 3 & CAV            & 34 & C2\\
10/17/25 & S2  & 6:17 am  & 10:33 am  & 10:36 am & 3 & CAV            & 33 & C2\\
10/17/25 & S3  & 6:21 am  & 10:36 am  & 10:39 am & 3 & Authentication & 32 & C2\\
10/17/25 & S4  & 6:27 am  & 10:39 am  & 10:41 am & 2 & CAV            & 31 & C2\\
10/17/25 & S5  & 6:56 am  & 10:41 am  & 10:43 am & 2 & CAV            & 30 & C2\\
10/17/25 & S6  & 7:07 am  & 10:43 am  & 10:45 am & 2 & Diploma        & 29 & C2\\
10/17/25 & S7  & 7:09 am  & 10:45 am  & 10:47 am & 2 & Diploma        & 28 & C2\\
10/17/25 & S8  & 7:12 am  & 10:47 am  & 10:49 am & 2 & CAV            & 27 & C2\\
10/17/25 & S9  & 7:23 am  & 10:49 am  & 10:51 am & 2 & CAV            & 26 & C2\\
10/17/25 & S10 & 7:27 am  & 10:51 am  & 10:53 am & 2 & CAV            & 25 & C2\\
10/17/25 & S11 & 7:40 am  & 10:53 am  & 10:55 am & 2 & Diploma        & 24 & C2\\
10/17/25 & S12 & 7:43 am  & 10:55 am  & 10:57 am & 2 & TOR            & 23 & C2\\
10/17/25 & S13 & 7:59 am  & 10:57 am  & 11:00 am & 3 & TOR            & 22 & C2\\
10/17/25 & S14 & 8:01 am  & 12:00 pm  & 12:03 pm & 3 & CAV            & 21 & C2\\
10/17/25 & S15 & 8:07 am  & 12:03 pm  & 12:06 pm & 3 & TOR            & 20 & C2\\
10/17/25 & S16 & 8:16 am  & 12:06 pm  & 12:09 pm & 3 & CAV            & 19 & C2\\
10/17/25 & S17 & 8:27 am  & 12:09 pm  & 12:12 pm & 3 & TOR            & 18 & C2\\
10/17/25 & S18 & 8:38 am  & 12:12 pm  & 12:14 pm & 2 & CAV            & 17 & C2\\
10/17/25 & S19 & 8:48 am  & 12:14 pm  & 12:16 pm & 2 & CAV            & 16 & C2\\
10/17/25 & S20 & 8:49 am  & 12:16 pm  & 12:18 pm & 2 & CAV            & 15 & C2\\
10/17/25 & S21 & 8:53 am  & 12:18 pm  & 12:21 pm & 3 & Diploma        & 14 & C2\\ \hline
\end{tabular}%
}
\end{table}



\begin{table}[ht]
\centering
\label{tbl:sampleTbl4}
\resizebox{\textwidth}{!}{%
\begin{tabular}{c|c|c|c|c|c|c|c|c}
\hline
\textbf{Date} & \textbf{Student ID} & \textbf{Arrival Time} &
\textbf{Service Start} & \textbf{Service End} &
\textbf{Service Duration(min)} & \textbf{Purpose} & \textbf{Queue Length} & \textbf{Counter ID (COE)}\\ \hline
10/13/25 & S1  & 6:34 am  & 8:00 am  & 8:02 am  & 2 & COE & 8 & C3\\
10/13/25 & S2  & 6:38 am  & 8:02 am  & 8:04 am  & 2 & COE & 7 & C3\\
10/13/25 & S3  & 6:46 am  & 8:04 am  & 8:07 am  & 3 & COE & 6 & C3\\
10/13/25 & S4  & 6:53 am  & 8:07 am  & 8:09 am  & 2 & COE & 5 & C3\\
10/13/25 & S5  & 7:26 am  & 8:09 am  & 8:11 am  & 2 & COE & 4 & C3\\
10/13/25 & S6  & 7:36 am  & 8:11 am  & 8:13 am  & 2 & COE & 3 & C3\\
10/13/25 & S7  & 7:49 am  & 8:13 am  &  8:15 am & 2 & COE & 2 & C3\\
10/13/25 & S8  & 7:56 am  & 8:15 am  & 8:17 am  & 2 & COE & 1 & C3\\
10/13/25 & S9  & 8:12 am  & 8:17 am  & 8:20 am  & 3 & COE & 0 & C3\\
10/13/25 & S10 & 8:19 am  & 8:20 am  & 8:22 am  & 2 & COE & 0 & C3\\
10/13/25 & S11 & 8:59 am  & 9:00 am  & 9:02 am  & 2 & COE & 0 & C3\\
10/13/25 & S12 & 9:18 am  & 9:19 am  & 9:21 am  & 2 & COE & 0 & C3\\
10/13/25 & S13 & 9:48 am  & 9:49 am  & 9:52 am  & 3 & COE & 0 & C3\\
10/13/25 & S14 & 10:02 am & 10:03 am & 10:06 am & 3 & COE & 0 & C3\\
10/13/25 & S15 & 10:16 am & 10:17 am & 10:19 am & 2 & COE & 0 & C3\\
10/14/25 & S1  & 7:26 am  & 8:02 am  & 8:04 am  & 2 & COE & 0 & C3\\
10/14/25 & S2  & 7:38 am  & 8:04 am  & 8:06 am  & 2 & COE & 1 & C3\\
10/14/25 & S4  & 8:12 am  & 8:13 am  & 8:15 am  & 2 & COE & 0 & C3\\
10/14/25 & S5  & 8:27 am  & 8:28 am  & 8:29 am  & 1 & COE & 0 & C3\\
10/14/25 & S6  & 8:36 am  & 8:37 am  & 8:38 am  & 1 & COE & 0 & C3\\
10/14/25 & S7  & 8:39 am  & 8:40 am  & 8:41 am  & 1 & COE & 0 & C3\\
10/14/25 & S8  & 8:47 am  & 8:48 am  & 8:50 am  & 2 & COE & 0 & C3\\
10/14/25 & S9  & 8:54 am  & 8:55 am  & 8:56 am  & 1 & COE & 0 & C3\\
10/14/25 & S10 & 9:26 am  & 9:27 am  & 9:28 am  & 1 & COE & 0 & C3\\
10/14/25 & S11 & 9:49 am  & 9:50 am  & 9:51 am  & 1 & COE & 0 & C3\\
10/14/25 & S12 & 9:54 am  & 9:55 am  & 9:57 am  & 2 & COE & 0 & C3\\
10/14/25 & S13 & 10:00 am & 10:01 am & 10:02 am & 1 & COE & 0 & C3\\
10/14/25 & S14 & 10:36 am & 10:37 am & 10:39 am & 2 & COE & 0 & C3\\
10/15/25 & S1  & 7:16 am  & 8:01 am  & 8:03 am  & 2 & COE & 3 & C3\\
10/15/25 & S2  & 7:28 am  & 8:03 am  & 8:04 am  & 1 & COE & 2 & C3\\
10/15/25 & S3  & 7:49 am  & 8:04 am  & 8:06 am  & 2 & COE & 1 & C3\\
10/15/25 & S4  & 8:24 am  & 8:25 am  & 8:26 am  & 1 & COE & 0 & C3\\
10/15/25 & S5  & 9:29 am  & 9:30 am  & 9:31 am  & 1 & COE & 0 & C3\\
10/15/25 & S6  & 9:56 am  & 9:57 am  & 9:58 am  & 1 & COE & 0 & C3\\
10/15/25 & S7  & 11:04 am & 11:05 am & 11:06 am & 1 & COE & 0 & C3\\
10/15/25 & S8  & 11:23 am & 11:24 am & 11:25 am & 1 & COE & 0 & C3\\
10/15/25 & S9  & 11:48 am & 11:49 am & 11:50 am & 1 & COE & 0 & C3\\
10/15/25 & S10 & 1:02 pm  & 1:03 pm  & 1:05 pm  & 2 & COE & 0 & C3\\
10/15/25 & S11 & 1:35 pm  & 1:36 pm  & 1:37 pm  & 1 & COE & 0 & C3\\
10/15/25 & S12 & 1:37 pm  & 1:38 pm  & 1:40 pm  & 2 & COE & 1 & C3\\
10/15/25 & S13 & 1:49 pm  & 1:50 pm  & 1:52 pm  & 2 & COE & 0 & C3\\
10/15/25 & S14 & 2:16 pm  & 2:17 pm  & 2:19 pm  & 2 & COE & 0 & C3\\
10/16/25 & S1  & 7:27 am  & 8:00 am  & 8:02 am  & 2 & COE & 2 & C3\\
10/16/25 & S2  & 7:45 am  & 8:02 am  & 8:04 am  & 2 & COE & 1 & C3\\
10/16/25 & S3  & 8:45 am  & 8:46 am  & 8:47 am  & 1 & COE & 0 & C3\\
10/16/25 & S4  & 10:23 am & 10:24 am & 10:25 am & 1 & COE & 0 & C3\\
10/16/25 & S5  & 10:48 am & 10:49 am & 10:50 am & 1 & COE & 0 & C3\\
10/16/25 & S6  & 10:56 am & 10:57 am & 10:59 am & 2 & COE & 0 & C3\\
10/16/25 & S7  & 11:34 am & 11:35 am & 11:37 am & 2 & COE & 0 & C3\\
10/16/25 & S8  & 1:12 pm  & 1:13 pm  & 1:14 pm  & 1 & COE & 0 & C3\\
10/16/25 & S9  & 1:47 pm  & 1:48 pm  & 1:49 pm  & 1 & COE & 0 & C3\\
10/17/25 & S1  & 7:10 am  & 8:10 am  & 8:11 am  & 1 & COE & 4 & C3\\
10/17/25 & S2  & 7:26 am  & 8:11 am  & 8:12 am  & 1 & COE & 3 & C3\\
10/17/25 & S3  & 7:46 am  & 8:12 am  & 8:13 am  & 1 & COE & 2 & C3\\
10/17/25 & S4  & 7:59 am  & 8:13 am  & 8:14 am  & 1 & COE & 1 & C3\\
10/17/25 & S5  & 8:02 am  & 8:14 am  & 8:15 am  & 1 & COE & 0 & C3\\
10/17/25 & S6  & 8:45 am  & 8:46 am  & 8:48 am  & 2 & COE & 0 & C3\\
10/17/25 & S7  & 9:34 am  & 9:35 am  & 9:37 am  & 2 & COE & 0 & C3\\
10/17/25 & S8  & 10:34 am & 10:35 am & 10:37 am & 2 & COE & 0 & C3\\
10/17/25 & S9  & 10:56 am & 10:57 am & 10:59 am & 2 & COE & 0 & C3\\
10/17/25 & S10 & 1:05 pm  & 1:06 pm  & 1:07 pm  & 1 & COE & 0 & C3\\
10/17/25 & S11 & 1:07 pm  & 1:08 pm  & 1:09 pm  & 1 & COE & 0 & C3\\
10/17/25 & S12 & 1:34 pm  & 1:35 pm  & 1:36 pm  & 1 & COE & 0 & C3\\ \hline
\end{tabular}%
}
\end{table}


\begin{table}[ht]
\centering
\label{tbl:sampleTbl4}
\resizebox{\textwidth}{!}{%
\begin{tabular}{c|c|c|c|c|c|c|c|c}
\hline
\textbf{Date} & \textbf{Student ID} & \textbf{Arrival Time} &
\textbf{Service Start} & \textbf{Service End} &
\textbf{Service Duration(min)} & \textbf{Purpose} & \textbf{Queue Length} & \textbf{Counter ID (ID VALIDATION)}\\ \hline
10/13/25 & S1  & 6:37 am  & 8:00 am  & 8:02 am  & 2 & ID VALIDATION & 8  & C4\\
10/13/25 & S2  & 6:47 am  & 8:02 am  & 8:04 am  & 2 & ID VALIDATION & 7  & C4\\
10/13/25 & S3  & 6:58 am  & 8:04 am  & 8:05 am  & 1 & ID VALIDATION & 6  & C4\\
10/13/25 & S4  & 7:16 am  & 8:05 am  & 8:06 am  & 1 & ID VALIDATION & 5  & C4\\
10/13/25 & S5  & 7:27 am  & 8:06 am  & 8:07 am  & 1 & ID VALIDATION & 4  & C4\\
10/13/25 & S6  & 7:30 am  & 8:07 am  & 8:08 am  & 1 & ID VALIDATION & 3  & C4\\
10/13/25 & S7  & 7:37 am  & 8:08 am  & 8:09 am  & 1 & ID VALIDATION & 2  & C4\\
10/13/25 & S8  & 7:49 am  & 8:09 am  & 8:11 am  & 2 & ID VALIDATION & 1  & C4\\
10/13/25 & S9  & 8:25 am  & 8:26 am  & 8:28 am  & 2 & ID VALIDATION & 0  & C4\\
10/13/25 & S10 & 8:40 am  & 8:41 am  & 8:43 am  & 2 & ID VALIDATION & 0  & C4\\
10/13/25 & S11 & 9:12 am  & 9:13 am  & 9:14 am  & 1 & ID VALIDATION & 0  & C4\\
10/13/25 & S13 & 9:47 am  & 9:48 am  & 9:50 am  & 2 & ID VALIDATION & 0  & C4\\
10/13/25 & S14 & 9:50 am  & 9:51 am  & 9:52 am  & 1 & ID VALIDATION & 0  & C4\\
10/13/25 & S15 & 10:03 am & 10:04 am & 10:05 am & 1 & ID VALIDATION & 0  & C4\\
10/13/25 & S16 & 10:05 am & 10:06 am & 10:08 am & 2 & ID VALIDATION & 0  & C4\\
10/13/25 & S17 & 10:10 am & 10:11 am & 10:13 am & 2 & ID VALIDATION & 0  & C4\\
10/13/25 & S18 & 10:37 am & 10:37 am & 10:38 am & 1 & ID VALIDATION & 0  & C4\\
10/13/25 & S19 & 10:38 am & 10:38 am & 10:40 am & 2 & ID VALIDATION & 0  & C4\\
10/13/25 & S20 & 10:40 am & 10:40 am & 10:41 am & 1 & ID VALIDATION & 0  & C4\\
10/14/25 & S1  & 6:19 am  & 8:05 am  & 8:06 am  & 1 & ID VALIDATION & 6  & C4\\
10/14/25 & S2  & 6:27 am  & 8:06 am  & 8:08 am  & 2 & ID VALIDATION & 5  & C4\\
10/14/25 & S3  & 6:48 am  & 8:08 am  & 8:10 am  & 2 & ID VALIDATION & 4  & C4\\
10/14/25 & S4  & 6:56 am  & 8:10 am  & 8:12 am  & 2 & ID VALIDATION & 4  & C4\\
10/14/25 & S5  & 7:14 am  & 8:12 am  & 8:13 am  & 1 & ID VALIDATION & 3  & C4\\
10/14/25 & S6  & 7:37 am  & 8:13 am  & 8:14 am  & 1 & ID VALIDATION & 2  & C4\\
10/14/25 & S7  & 7:56 am  & 8:14 am  & 8:15 am  & 1 & ID VALIDATION & 1  & C4\\
10/14/25 & S8  & 8:10 am  & 8:15 am  & 8:16 am  & 1 & ID VALIDATION & 0  & C4\\
10/14/25 & S9  & 8:36 am  & 8:37 am  & 8:38 am  & 1 & ID VALIDATION & 0  & C4\\
10/14/25 & S10 & 8:39 am  & 8:40 am  & 8:42 am  & 2 & ID VALIDATION & 0  & C4\\
10/14/25 & S11 & 8:48 am  & 8:49 am  & 8:50 am  & 1 & ID VALIDATION & 0  & C4\\
10/14/25 & S12 & 8:56 am  & 8:57 am  & 8:59 am  & 2 & ID VALIDATION & 0  & C4\\
10/14/25 & S13 & 9:03 am  & 9:04 am  & 9:05 am  & 1 & ID VALIDATION & 0  & C4\\
10/14/25 & S14 & 9:35 am  & 9:36 am  & 9:37 am  & 1 & ID VALIDATION & 0  & C4\\
10/14/25 & S15 & 9:47 am  & 9:48 am  & 9:49 am  & 1 & ID VALIDATION & 0  & C4\\
10/14/25 & S16 & 10:01 am & 10:02 am & 10:04 am & 2 & ID VALIDATION & 0  & C4\\
10/15/25 & S1  & 6:34 am  & 8:00 am  & 8:01 am  & 1 & ID VALIDATION & 7  & C4\\
10/15/25 & S2  & 6:46 am  & 8:01 am  & 8:02 am  & 1 & ID VALIDATION & 6  & C4\\
10/15/25 & S3  & 6:48 am  & 8:02 am  & 8:03 am  & 1 & ID VALIDATION & 5  & C4\\
10/15/25 & S4  & 6:58 am  & 8:03 am  & 8:04 am  & 1 & ID VALIDATION & 4  & C4\\
10/15/25 & S5  & 7:00 am  & 8:04 am  & 8:05 am  & 1 & ID VALIDATION & 3  & C4\\
10/15/25 & S6  & 7:46 am  & 8:05 am  & 8:07 am  & 2 & ID VALIDATION & 2  & C4\\
10/15/25 & S7  & 7:52 am  & 8:07 am  & 8:09 am  & 2 & ID VALIDATION & 1  & C4\\
10/15/25 & S8  & 8:05 am  & 8:09 am  & 8:11 am  & 2 & ID VALIDATION & 0  & C4\\
10/15/25 & S9  & 8:16 am  & 8:17 am  & 8:18 am  & 1 & ID VALIDATION & 0  & C4\\
10/15/25 & S10 & 8:18 am  & 8:19 am  & 8:21 am  & 2 & ID VALIDATION & 0  & C4\\
10/15/25 & S11 & 8:24 am  & 8:25 am  & 8:26 am  & 1 & ID VALIDATION & 0  & C4\\
10/15/25 & S12 & 9:15 am  & 9:16 am  & 9:18 am  & 2 & ID VALIDATION & 0  & C4\\
10/15/25 & S13 & 9:19 am  & 9:19 am  & 9:20 am  & 1 & ID VALIDATION & 0  & C4\\
10/15/25 & S14 & 9:28 am  & 9:29 am  & 9:31 am  & 2 & ID VALIDATION & 0  & C4\\
10/15/25 & S15 & 9:34 am  & 9:35 am  & 9:37 am  & 2 & ID VALIDATION & 0  & C4\\
10/15/25 & S16 & 9:40 am  & 9:41 am  & 9:42 am  & 1 & ID VALIDATION & 0  & C4\\
10/15/25 & S17 & 9:59 am  & 10:00 am & 10:01 am & 1 & ID VALIDATION & 0  & C4\\
10/15/25 & S18 & 10:04 am & 10:05 am & 10:06 am & 1 & ID VALIDATION & 0  & C4\\
10/15/25 & S19 & 10:16 am & 10:17 am & 10:18 am & 1 & ID VALIDATION & 0  & C4\\
10/15/25 & S20 & 10:45 am & 10:46 am & 10:47 am & 1 & ID VALIDATION & 0  & C4\\
10/15/25 & S21 & 10:57 am & 10:58 am & 10:59 am & 1 & ID VALIDATION & 0  & C4\\
10/15/25 & S22 & 11:16 am & 11:17 am & 11:19 am & 2 & ID VALIDATION & 0  & C4\\
10/15/25 & S23 & 11:46 am & 11:47 am & 11:48 am & 1 & ID VALIDATION & 0  & C4\\
10/15/25 & S24 & 11:56 am & 11:57 am & 11:58 am & 1 & ID VALIDATION & 0  & C4\\
10/15/25 & S25 & 1:12 pm  & 1:13 pm  & 1:14 pm  & 1 & ID VALIDATION & 0  & C4\\
10/16/25 & S1  & 7:21 am  & 8:10 am  & 8:12 am  & 2 & ID VALIDATION & 3  & C4\\
10/16/25 & S2  & 7:29 am  & 8:18 am  & 8:19 am  & 1 & ID VALIDATION & 2  & C4\\
10/16/25 & S3  & 7:44 am  & 8:45 am  & 8:46 am  & 1 & ID VALIDATION & 1  & C4\\
10/16/25 & S4  & 8:01 am  & 8:46 am  & 8:47 am  & 1 & ID VALIDATION & 0  & C4\\
10/16/25 & S5  & 8:28 am  & 8:47 am  & 8:48 am  & 1 & ID VALIDATION & 0  & C4\\
10/16/25 & S6  & 8:45 am  & 8:48 am  & 8:50 am  & 2 & ID VALIDATION & 1  & C4\\
10/16/25 & S7  & 9:15 am  & 9:16 am  & 9:18 am  & 2 & ID VALIDATION & 0  & C4\\
10/16/25 & S8  & 9:27 am  & 9:28 am  & 9:30 am  & 2 & ID VALIDATION & 0  & C4\\
10/16/25 & S9  & 9:56 am  & 9:57 am  & 9:58 am  & 1 & ID VALIDATION & 0  & C4\\
10/16/25 & S10 & 10:02 am & 10:03 am & 10:04 am & 1 & ID VALIDATION & 0  & C4\\
10/16/25 & S11 & 10:18 am & 10:19 am & 10:21 am & 2 & ID VALIDATION & 0  & C4\\
10/17/25 & S1  & 6:48 am  & 8:30 am  & 8:32 am  & 2 & ID VALIDATION & 6  & C4\\
10/17/25 & S2  & 6:56 am  & 8:32 am  & 8:34 am  & 2 & ID VALIDATION & 5  & C4\\
10/17/25 & S3  & 7:12 am  & 8:34 am  & 8:36 am  & 2 & ID VALIDATION & 4  & C4\\
10/17/25 & S4  & 7:16 am  & 8:36 am  & 8:38 am  & 2 & ID VALIDATION & 3  & C4\\
10/17/25 & S5  & 7:34 am  & 8:38 am  & 8:40 am  & 2 & ID VALIDATION & 2  & C4\\
10/17/25 & S6  & 7:56 am  & 8:40 am  & 8:41 am  & 1 & ID VALIDATION & 1  & C4\\
10/17/25 & S7  & 8:37 am  & 8:41 am  & 8:42 am  & 1 & ID VALIDATION & 0  & C4\\
10/17/25 & S8  & 8:57 am  & 8:58 am  & 9:00 am  & 2 & ID VALIDATION & 0  & C4\\
10/17/25 & S9  & 9:34 am  & 9:35 am  & 9:36 am  & 1 & ID VALIDATION & 0  & C4\\
10/17/25 & S10 & 9:48 am  & 9:49 am  & 9:50 am  & 1 & ID VALIDATION & 0  & C4\\
10/17/25 & S11 & 9:58 am  & 9:59 am  & 10:00 am & 1 & ID VALIDATION & 0  & C4\\
10/17/25 & S12 & 10:00 am & 10:00 am & 10:02 am & 2 & ID VALIDATION & 0  & C4\\ \hline
\end{tabular}%
}
\end{table}	 

\chapter{Non-Disclosure Agreement Form(NDA)}

\begin{figure}[htb]
	\centering	\includegraphics[width=0.85\linewidth]{figures/Tiffany_1.jpg}
\end{figure}

\begin{figure}[htb]
	\centering
	\includegraphics[width=\linewidth]{figures/Tiffany_2.jpg}
\end{figure}

\begin{figure}[htb]
	\centering
	\includegraphics[width=\linewidth]{figures/Freddie_1.jpg}
\end{figure}

\begin{figure}[htb]
	\centering
	\includegraphics[width=\linewidth]{figures/Freddie_2.jpg}
\end{figure}

\begin{figure}[htb]
	\centering
	\includegraphics[width=\linewidth]{figures/Oñate_1.jpg}
\end{figure}

\begin{figure}[htb]
	\centering
	\includegraphics[width=\linewidth]{figures/Oñate_2.jpg}
\end{figure}

\chapter{JOINT AFFIDAVIT OF UNDERTAKING (Plagiarism)}
\begin{figure}[htb]
	\centering
	\includegraphics[width=\linewidth]{figures/affidavit_1.jpg}
\end{figure}

\chapter{Project Team Assignment Form}

\chapter{Role Acceptance Form}
\begin{figure}[htb]
	\centering
	\includegraphics[width=0.85\linewidth]{figures/Kaela_1.jpg}
\end{figure}

\begin{figure}[htb]
	\centering
	\includegraphics[width=\linewidth]{figures/Juan_2.jpg}
\end{figure}

\begin{figure}[htb]
	\centering
	\includegraphics[width=\linewidth]{figures/Jayvie_3.jpg}
\end{figure}

\chapter{Final Project Title Form}

\begin{figure}[htb]
	\centering
	\includegraphics[width=0.85\linewidth]{figures/Project_1.jpg}
\end{figure}

\chapter{Thesis/Capstone Project Hearing Form}

\begin{figure}[htb]
	\centering
	\includegraphics[width=0.85\linewidth]{figures/HearingFormTD_1.jpg}
\end{figure}

\begin{figure}[htb]
	\centering
	\includegraphics[width=\linewidth]{figures/HearingPOD_2.jpg}
\end{figure}

\chapter{Recommendations, Suggestions and Comment (RSC)}

\begin{figure}[htb]
	\centering
	\includegraphics[width=0.85\linewidth]{figures/Recommend.jpg}
\end{figure}

\chapter{Letter for Data Gathering}
\begin{figure}[htb]
	\centering
	\includegraphics[width=0.85\linewidth]{figures/Letter.jpg}
\end{figure}

\chapter{Consultation Log Forms}
\begin{figure}[htb]
	\centering
	\includegraphics[width=0.85\linewidth]{figures/Consultation_1.jpg}
\end{figure}
\begin{figure}[htb]
	\centering
	\includegraphics[width=\linewidth]{figures/Consultation_2.jpg}
\end{figure}
\begin{figure}[htb]
	\centering
	\includegraphics[width=\linewidth]{figures/Consultation_3.jpg}
\end{figure}
\begin{figure}[htb]
	\centering
	\includegraphics[width=\linewidth]{figures/Consultation_4.jpg}
\end{figure}
\begin{figure}[htb]
	\centering
	\includegraphics[width=\linewidth]{figures/Consultation_5.jpg}
\end{figure}	  
\begin{figure}[htb]
	\centering
	\includegraphics[width=\linewidth]{figures/Consultation_6.jpg}
\end{figure}
\begin{figure}[htb]
	\centering
	\includegraphics[width=\linewidth]{figures/Consultation_7.jpg}
\end{figure}
\begin{figure}[htb]
	\centering
	\includegraphics[width=\linewidth]{figures/Consultation_8.jpg}
\end{figure}
\begin{figure}[htb]
	\centering
	\includegraphics[width=\linewidth]{figures/Consultation_9.jpg}
\end{figure}
\begin{figure}[htb]
	\centering
	\includegraphics[width=\linewidth]{figures/Consultation_10.jpg}
\end{figure}
\begin{figure}[htb]
	\centering
	\includegraphics[width=\linewidth]{figures/Consultation_11.jpg}
\end{figure}
\begin{figure}[htb]
	\centering
	\includegraphics[width=\linewidth]{figures/Consultation_12.jpg}
\end{figure}
\begin{figure}[htb]
	\centering
	\includegraphics[width=\linewidth]{figures/Consultation_13.jpg}
\end{figure}


\chapter{Language Editing Certification}
\centering

This is to certify that the undersigned has reviewed and went through all the pages of the Bachelor of Science in Computer Science thesis manuscript titled \\

\textbf{"COMPARATIVE ANALYSIS OF DISCRETE-EVENT AND CONTINUOUS SIMULATION MODELS FOR STUDENTS FLOW OPTIMIZATION"} \\

\textbf{May Angeline M. Regaspi}, \textbf{Abegail C. Repia}, \textbf{Joy Ann M. Ruin}, as against the set of structural rules that govern research writing in accord with the composition of sentences, phrases, and words in the English language.
 \newline \newline \newline \\

\noindent \textbf{MS. JAYVIE B. MARGATE} \\
\textit{Language Editor} \\

Date:\_\_\_\_\_\_\_\_\_\_\_\_\_\_\_\_\_\_\_\_\_\_\_


\chapter{Secretary's Certification}
\centering

This is to certify that the undersigned has provided accurate recommendations, suggestions, and comments unanimously agreed and approved by the panel of examiners during the oral examination of the thesis titled \\ \textbf{"COMPARATIVE ANALYSIS OF DISCRETE-EVENT AND CONTINUOUS SIMULATION MODELS FOR STUDENTS FLOW OPTIMIZATION"} \\  prepared and submitted by \textbf{May Angeline M. Regaspi}, \textbf{Abegail C. Repia}, \textbf{Joy Ann M. Ruin}, and that the same have not been amended, modified or obliterated. \newline \newline \newline \\



\textbf{MS. MARRI GRACE MORATA} \\
\textit{Secretary} \\


Date:\_\_\_\_\_\_\_\_\_\_\_\_\_\_\_\_\_\_\_\_\_\_\_

\end{theappendices}

    % Vita should only be included for PhD candidates.

\begin{vita}

\begin{itemize}
    \item 
    
    \begin{figure}[ht]
        \centering
    	\includegraphics[width=0.35\textwidth]{figures/person-icon.png}
    \end{figure}
    
    \textbf{Joseph Jessie S. Oñate} is a faculty member of the College of Computer Studies. He finished his Master of Science in Computer Science degree at Ateneo de Naga University. His research interests focused on Intelligent Systems, Algorithm and Complexity, Web Technologies, Computer Vision, and Graphics.
    
    \item 
    
    \begin{figure}[ht]
        \centering
    	\includegraphics[width=0.35\textwidth]{figures/person-icon.png}
    \end{figure}
    
    \textbf{Joseph Jessie S. Oñate} is a faculty member of the College of Computer Studies. He finished his Master of Science in Computer Science degree at Ateneo de Naga University. His research interests focused on Intelligent Systems, Algorithm and Complexity, Web Technologies, Computer Vision, and Graphics.
    
    \item 
    
    \begin{figure}[H]
        \centering
    	\includegraphics[width=0.35\textwidth]{figures/person-icon.png}
    \end{figure}
    
    \textbf{Joseph Jessie S. Oñate} is a faculty member of the College of Computer Studies. He finished his Master of Science in Computer Science degree at Ateneo de Naga University. His research interests focused on Intelligent Systems, Algorithm and Complexity, Web Technologies, Computer Vision, and Graphics.
\end{itemize}

\end{vita}
\end{thesisbody}

\end{document}